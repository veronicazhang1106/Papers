% Options for packages loaded elsewhere
\PassOptionsToPackage{unicode}{hyperref}
\PassOptionsToPackage{hyphens}{url}
\PassOptionsToPackage{dvipsnames,svgnames,x11names}{xcolor}
%
\documentclass[
  authoryear]{elsarticle}

\usepackage{amsmath,amssymb}
\usepackage{iftex}
\ifPDFTeX
  \usepackage[T1]{fontenc}
  \usepackage[utf8]{inputenc}
  \usepackage{textcomp} % provide euro and other symbols
\else % if luatex or xetex
  \usepackage{unicode-math}
  \defaultfontfeatures{Scale=MatchLowercase}
  \defaultfontfeatures[\rmfamily]{Ligatures=TeX,Scale=1}
\fi
\usepackage{lmodern}
\ifPDFTeX\else  
    % xetex/luatex font selection
\fi
% Use upquote if available, for straight quotes in verbatim environments
\IfFileExists{upquote.sty}{\usepackage{upquote}}{}
\IfFileExists{microtype.sty}{% use microtype if available
  \usepackage[]{microtype}
  \UseMicrotypeSet[protrusion]{basicmath} % disable protrusion for tt fonts
}{}
\makeatletter
\@ifundefined{KOMAClassName}{% if non-KOMA class
  \IfFileExists{parskip.sty}{%
    \usepackage{parskip}
  }{% else
    \setlength{\parindent}{0pt}
    \setlength{\parskip}{6pt plus 2pt minus 1pt}}
}{% if KOMA class
  \KOMAoptions{parskip=half}}
\makeatother
\usepackage{xcolor}
\setlength{\emergencystretch}{3em} % prevent overfull lines
\setcounter{secnumdepth}{5}
% Make \paragraph and \subparagraph free-standing
\makeatletter
\ifx\paragraph\undefined\else
  \let\oldparagraph\paragraph
  \renewcommand{\paragraph}{
    \@ifstar
      \xxxParagraphStar
      \xxxParagraphNoStar
  }
  \newcommand{\xxxParagraphStar}[1]{\oldparagraph*{#1}\mbox{}}
  \newcommand{\xxxParagraphNoStar}[1]{\oldparagraph{#1}\mbox{}}
\fi
\ifx\subparagraph\undefined\else
  \let\oldsubparagraph\subparagraph
  \renewcommand{\subparagraph}{
    \@ifstar
      \xxxSubParagraphStar
      \xxxSubParagraphNoStar
  }
  \newcommand{\xxxSubParagraphStar}[1]{\oldsubparagraph*{#1}\mbox{}}
  \newcommand{\xxxSubParagraphNoStar}[1]{\oldsubparagraph{#1}\mbox{}}
\fi
\makeatother


\providecommand{\tightlist}{%
  \setlength{\itemsep}{0pt}\setlength{\parskip}{0pt}}\usepackage{longtable,booktabs,array}
\usepackage{calc} % for calculating minipage widths
% Correct order of tables after \paragraph or \subparagraph
\usepackage{etoolbox}
\makeatletter
\patchcmd\longtable{\par}{\if@noskipsec\mbox{}\fi\par}{}{}
\makeatother
% Allow footnotes in longtable head/foot
\IfFileExists{footnotehyper.sty}{\usepackage{footnotehyper}}{\usepackage{footnote}}
\makesavenoteenv{longtable}
\usepackage{graphicx}
\makeatletter
\newsavebox\pandoc@box
\newcommand*\pandocbounded[1]{% scales image to fit in text height/width
  \sbox\pandoc@box{#1}%
  \Gscale@div\@tempa{\textheight}{\dimexpr\ht\pandoc@box+\dp\pandoc@box\relax}%
  \Gscale@div\@tempb{\linewidth}{\wd\pandoc@box}%
  \ifdim\@tempb\p@<\@tempa\p@\let\@tempa\@tempb\fi% select the smaller of both
  \ifdim\@tempa\p@<\p@\scalebox{\@tempa}{\usebox\pandoc@box}%
  \else\usebox{\pandoc@box}%
  \fi%
}
% Set default figure placement to htbp
\def\fps@figure{htbp}
\makeatother

\usepackage{booktabs}
\usepackage{longtable}
\usepackage{array}
\usepackage{multirow}
\usepackage{wrapfig}
\usepackage{float}
\usepackage{colortbl}
\usepackage{pdflscape}
\usepackage{tabu}
\usepackage{threeparttable}
\usepackage{threeparttablex}
\usepackage[normalem]{ulem}
\usepackage{makecell}
\usepackage{xcolor}
\usepackage{graphicx}
\usepackage{adjustbox}
\makeatletter
\@ifpackageloaded{caption}{}{\usepackage{caption}}
\AtBeginDocument{%
\ifdefined\contentsname
  \renewcommand*\contentsname{Table of contents}
\else
  \newcommand\contentsname{Table of contents}
\fi
\ifdefined\listfigurename
  \renewcommand*\listfigurename{List of Figures}
\else
  \newcommand\listfigurename{List of Figures}
\fi
\ifdefined\listtablename
  \renewcommand*\listtablename{List of Tables}
\else
  \newcommand\listtablename{List of Tables}
\fi
\ifdefined\figurename
  \renewcommand*\figurename{Figure}
\else
  \newcommand\figurename{Figure}
\fi
\ifdefined\tablename
  \renewcommand*\tablename{Table}
\else
  \newcommand\tablename{Table}
\fi
}
\@ifpackageloaded{float}{}{\usepackage{float}}
\floatstyle{ruled}
\@ifundefined{c@chapter}{\newfloat{codelisting}{h}{lop}}{\newfloat{codelisting}{h}{lop}[chapter]}
\floatname{codelisting}{Listing}
\newcommand*\listoflistings{\listof{codelisting}{List of Listings}}
\makeatother
\makeatletter
\makeatother
\makeatletter
\@ifpackageloaded{caption}{}{\usepackage{caption}}
\@ifpackageloaded{subcaption}{}{\usepackage{subcaption}}
\makeatother
\journal{The British Accounting Review}

\usepackage[]{natbib}
\bibliographystyle{elsarticle-harv}
\usepackage{bookmark}

\IfFileExists{xurl.sty}{\usepackage{xurl}}{} % add URL line breaks if available
\urlstyle{same} % disable monospaced font for URLs
\hypersetup{
  pdftitle={20250521-CSP\_CFP literature review},
  pdfauthor={Lisa Sheenan; Atul Saxena; Ying (Veronica) Zhang},
  pdfkeywords={ESG disclosure, IFRS S1 and S2, Stakeholder-Legitimacy
framework, bank risk, ownership, climate change},
  colorlinks=true,
  linkcolor={blue},
  filecolor={Maroon},
  citecolor={Blue},
  urlcolor={Blue},
  pdfcreator={LaTeX via pandoc}}


\setlength{\parindent}{6pt}
\begin{document}
\begin{frontmatter}
\title{20250521-CSP\_CFP literature review}

\author[1]{Lisa Sheenan%
\corref{cor1}%
}
 \ead{lisa.sheenan@ucd.ie} 
\author[2]{Atul Saxena%
%
}
 \ead{asaxena@ggc.edu} 
\author[3]{Ying (Veronica) Zhang%
%
}
 \ead{veronica.zhang@qub.ac.uk} 
\affiliation[1]{organization={University College Dublin, UCD College of
Business},addressline={Dublin, Ireland},postcodesep={}}
\affiliation[2]{organization={Georgia Gwinnett
College},addressline={Georgia, US},postcodesep={}}
\affiliation[3]{organization={Queen's University, Belfast, Queen's
Business School},addressline={Belfast, UK},postcodesep={}}

\cortext[cor1]{Corresponding author}



        
\begin{abstract}
The following sections review the theoretical frameworks of the
relationship between ESG reporting and financial institutions. The
theoretical frameworks include stakeholder theory, legitimacy theory,
and institutional theory. The study also reviews emerging frameworks
such as natural capital and double materiality. The review proposes a
Stakeholder-Legitimacy Framework, integrating stakeholder theory,
legitimacy theory, and institutional isomorphism to explain how banks'
ESG disclosures emerge from dynamic negotiations between external
pressures (e.g., regulators, investors) and strategic legitimacy-seeking
behaviors. Key findings from the review of empirical research reveal
that while ESG disclosures mitigate bank-specific risks, conflicting
evidence persists on their financial performance impacts. Critical gaps
include underexplored transmission effects of regulations through
lending networks, inconsistent roles of institutional and state
ownership, and tensions between symbolic compliance and substantive
action. The research question can be: What is the relationship between
ESG reporting and banks? How does ESG reporting regulation impact bank
performance, for example, the EBA's Green Asset Ratios? Using the
integrated stakeholder-legitimacy framework, we can examine the
influencing factors such as regulation and bank performance.
\end{abstract}





\begin{keyword}
    ESG disclosure \sep IFRS S1 and S2 \sep Stakeholder-Legitimacy
framework \sep bank risk \sep ownership \sep 
    climate change
\end{keyword}
\end{frontmatter}
    

\section{Theoretical Foundation}\label{theoretical-foundation}

The relationship between environmental, social, and governance (ESG)
reporting and financial institutions has garnered increasing scholarly
attention, driven by global sustainability challenges. This section
first establishes the conceptual landscape of contemporary ESG
disclosure before synthesizes theoretical frameworks that explain how
and why ESG reporting influences financial institutions' operations,
decision-making, and market outcomes.

\subsection{Conceptual Advances in ESG
Disclosure}\label{conceptual-advances-in-esg-disclosure}

Contemporary theoretical concepts and analytical lenses around natural
capital and double materiality offer novel perspectives on integrating
environmental considerations into financial decision-making. ESG
reporting is not merely a compliance exercise but a strategic imperative
for financial institutions. Natural capital concept, as articulated by
\citet{DASGUPTA2021} and \citet{DAILY2009}, emphasizes the valuation of
biodiversity and natural capital, with implications for financial
modeling \citep{ATKINSON2014}. \citet{ECCLES2014} provide a theoretical
foundation for integrated reporting, which aligns closely with double
materiality. The concept of ``double-materiality'' was first formally
proposed by the European Commission \citep{EC2019} in Guidelines on
Non-financial Reporting: Supplement on Reporting Climate-related
Information published in June 2019; then was formalized by the
regulation of the Corporate Sustainability Reporting Directive
(CSRD)\citep[see][]{EU_CSRD_2022} as a reporting requirement. It
mandates a company to judge materiality from two perspectives 1) ``the
extent necessary for an understanding of the company's development,
performance and position'' and ``in the broad sense of affecting the
value of the company''; 2) environmental and social impact of the
company's activities on a broad range of stakeholders.

\subsection{Organizational Theories}\label{organizational-theories}

Organizational theories, particularly stakeholder theory and legitimacy
theory, highlight the institutional motivations and external pressures
driving ESG reporting practices. Stakeholder theory, developed by
\citet{FREEMAN1984}, posits that organizations are accountable to a
broad range of stakeholders beyond shareholders, including regulators,
customers (investors), employees, and civil society. Building on this,
\citet{MITCHELL1997} proposed the theory of stakeholder salience
emphasizing understanding why and how managers prioritize certain
stakeholders, rather than prescribing whom they should prioritize. Their
theory provides tools to identify stakeholder types via a typology and
explain managerial responses through the concept of salience, which
assesses stakeholders' legitimacy, power, and urgency. This typology
offers a framework for financial institutions to navigate competing
stakeholder interests effectively. \citet{DONALDSON1995} provided
theoretical justification for stakeholder theory, emphasizing its
descriptive, instrumental, and normative validity in explaining
corporate behavior. \citet{ROBERTS1992} conducts an application of
stakeholder theory to explain corporate social responsibility
disclosure. The study finds that measures of stakeholder power,
strategic posture, and economic performance are significantly associated
with levels of corporate social disclosure. Banks' ESG disclosure, for
instance, on climate risks or green lending portfolios are often framed
as responses to investor demands for transparency or regulatory mandates
like the EU Taxonomy and the EU's Corporate Sustainability Reporting
Directive (CSRD).

Legitimacy theory, foundationally developed by \citet{SUCHMAN1995},
identifies three primary forms of legitimacy\footnote{Legitimacy is a
  generalized perception or assumption that the actions of an entity are
  desirable, proper, or appropriate within some socially constructed
  system of norms, values, beliefs, and definitions
  \citep{GINZEL2004, NEILSEN1987, PERROW1970}.}---pragmatic, moral, and
cognitive---that organizations seek to maintain. \citet{DEEGAN2002}
applied this framework to environmental and social disclosures,
demonstrating how these practices reinforce an organization's
legitimacy. Firms' legitimacy-seeking behaviors influence the depth and
scope of disclosures, with firms in environmentally sensitive industries
disclosing more to manage public perception. In banking contexts, ESG
disclosures function as legitimacy-management tools to address
stakeholder pressures, as evidenced by sector-specific studies
\citep[see][]{THOMPSON2004, MURE2021, CARNEVALE2014}.

Building on stakeholder theory and legitimacy theory, institutional
isomorphism elucidates the structured mechanisms that drive
organizational conformity. \citet{DIMAGGIO1983} identify three
isomorphic drivers that create organizational homogeneity:

\begin{enumerate}
\def\labelenumi{\arabic{enumi}.}
\item
  \textbf{Coercive isomorphism}: explains how regulatory requirements
  and pressure from stakeholders drive adoption of standardized
  reporting formats.
\item
  \textbf{Mimetic isomorphism}: illuminates why organizations copy
  ``best practice'' reporting methods from industry leaders during
  periods of uncertainty.
\item
  \textbf{Normative isomorphism}: shows how professional networks and
  training lead to similar reporting approaches across organizations.
\end{enumerate}

Their foundation study develops a theoretical framework explaining how
these mechanisms operate within organizational fields. This framework is
valuable in understanding the standardization and diffusion of reporting
practices across organizations. Empirical research highlights the
importance of institutional pressures, including coercive, normative,
and mimetic forces, in driving firms to conform to ESG expectations.
\citet{DELMAS2004} demonstrate firms' heterogeneous interpretations of
isomorphic forces determine substantive versus symbolic ESG integration.
This divergence mirrors banking practices where identical EU CSRD
mandates yield divergent TNFD implementation depth.
\citet{BEBBINGTON2018} position normative isomorphism as instrumental in
bridging accounting praxis with planetary boundaries, arguing that
professional networks institutionalize ``double materiality'' reporting,
which notes the field's underdevelopment in translating SDGs into bank
capital allocation. \citet{CHRISTENSEN2021} reveal mimetic isomorphism's
unintended consequences: firms under US/EU disclosure mandates
increasingly adopt boilerplate ESG language to signal conformity while
obscuring institution-specific ecological impacts.

\subsection{Economic Theories}\label{economic-theories}

Economic theories, particularly information asymmetry and agency theory,
explain how ESG reporting enhances market efficiency and reduces capital
costs. \citet{JENSEN1976} established the foundational agency theory
framework, explaining the conflicts of interest between principals
(shareholders) and agents (managers). This study lays the groundwork for
understanding how ESG reporting can mitigate agency costs by aligning
stakeholder interests. \citet{AKERLOF1970} work on Information Asymmetry
further explains how unequal information among market participants
creates inefficiencies. \citet{HART1987} provide a foundation for modern
contract theory and influence the theory of the firm and organizational
economics. This paper develops the principal-agent model, and
demonstrate how disclosure requirements can act as screening devices.
\citet{GRAY1996} provide a theoretical framework for understanding the
role of corporate social and environmental reporting in addressing
information asymmetry and enhancing accountability. Applying these
theories, studies have demonstrated that ESG disclosure reduces
information gaps and lowers agency costs, thereby enhancing market
efficiency. For instance, \citet{DHALIWAL2011} and \citet{ELGHOUL2011}
find firms that initiate CSR disclosure and have superior CSR
performance (relative to their industry peers) experience a subsequent
reduction in their cost of equity capital, and attract more dedicated
institutional investors and analyst coverage. Moreover, \citet{CUI2018}
observed that CSR engagement is inversely associated with reputation
risk, largely due to reduced information asymmetry.

\section{Extant Empirical Evidence}\label{extant-empirical-evidence}

\subsection{Firm Value and
Performance}\label{firm-value-and-performance}

Stakeholder theories \citep[see][]{FREEMAN1984, DEEGAN2002} suggest that
when organizations incorporate environmental, social, and governance
(ESG) practices into their long-term strategic planning, they gain
competitive advantages by addressing the needs of various groups - from
employees and customers to government bodies and community members. This
approach recognizes that business success is linked to creating value
for all stakeholders, not just shareholders. The stakeholders reward
companies with good ESG practices through investment, consumption and
higher productivity \citep{LI2018}In contrast, the trade-off hypothesis
or traditionalist view \citep[see][]{FRIEDMAN2007} suggests ESG
practices negatively impact financial performance by increasing costs,
harming corporate performance, and reducing competitive advantage.
Scholars argue that focusing on social and environmental goals diverts
managers from maximizing shareholder value. Similarly, satisfying
non-shareholder stakeholders may hinder profit maximization and value
creation for owners and managers\citep{GALANT2017}.

Empirical studies on the relationship between ESG engagement and
financial performance have provided conflicting findings. Driven by
stakeholder theory, \citet{WU2013} find that ESG disclosures improve
bank profitability by attracting ESG-conscious investors. Their findings
attribute to stakeholder pressure for transparency. Drawing from
empirical analysis using MSCI ESG STATS database ratings,
\citet{CORNETT2016} examine the relationship between the US commercial
banks' corporate social responsibility (CSR) practices and financial
performance. The study reveals that larger banks, despite facing
criticism for profit-focused practices leading to the crisis,
demonstrate consistently higher CSR strengths and concerns. Post-2009,
these institutions showed marked improvement with increased CSR
strengths and decreased concerns. Similarly, studies such as
\citet{CARNEVALE2014}, \citet{SHEN2016} and \citet{BUALLAY2021}
demonstrate banks' engagement in CSR activities can improve the
financial performance. \citet{TIAN2023} report that banks with green
credit announcement show better market value and long-term performance.
While \citet{FERRERO2016} find a nonlinear relationship between ESG and
financial performance employing listed companies in Europe. Moreover,
\citet{BUALLAY2023} investigate the effect of sustainability reporting
on bank performance (operational, financial and market) in 7 regions
that include 60 countries over the period 2008-2017. They find the
negative relationship between ESG disclosures and banks' operational
performance (ROA, ROE), and market performance (Tobin's Q). They argue
that sustainability reporting may have negative impact on banks' asset
utilization, in line with the trad-off theory \citep[see][]{LEE2009}.
\citet{ARAS2024} investigates the SDGs with ESG indicators through a
double materiality perspective for 1888 companies from the OECD
financial institutions. The study shows how commercial banks can
identify and prioritize the SDGs and targets and how sustainable
practices at the corporate level can contribute to achieving these
global goals by adopting a sound approach.

\subsection{Bank Risk}\label{bank-risk}

Despite growing scholarly attention to the link between CSR and bank
performance, empirical research examining its influence on risk exposure
within financial institutions remains limited. The existing literature
primarily focuses on individual bank risk metrics. A study by
\citet{GANGI2019} examine 142 banks across 35 countries and find that
stronger environmental engagement correlates with lower risk profiles as
measured by Z-scores \citep[see][]{LAEVEN2009}. Similarly,
\citet{SCHOLTENS2019} report a modest negative relationship between
sustainability practices and standalone risk measures among European
banks. Aligning with stakeholder theory, \citet{DI_TOMMASO2020}
demonstrate that higher ESG scores slightly curbed risk-taking among
European banks (2007--2018), though this relationship was mediated by
board composition factors including size, independent director presence,
and gender diversity. \citet{GANGWANI2024} report that higher ESG
disclosure scores correlate with reduced commercial banking risks,
specifically mitigating insolvency, leverage, and liquidity challenges.
Further evidence from \citet{CHIARAMONTE2022} demonstrate that both
aggregate ESG scores and individual pillars enhanced bank stability
during financial stress periods, with longer ESG disclosure histories
amplifying these stabilizing effects. Expanding geographically,
\citet{NEITZERT2022} examine 582 international banks (2002--2018),
corroborating the risk-mitigating effects of CSR activities on default
and portfolio risks, consistent with risk management frameworks.
However, their findings revealed heterogeneity across ESG pillars:
environmental (E) initiatives exhibited the strongest and most
consistent risk-reduction outcomes, whereas social (S) and governance
(G) dimensions yielded less definitive results.

Academic research on the relationship between systemic financial risk
and ESG factors remains limited, with existing studies examining each of
the three E, S and G pillars respectively. Regarding environmental (E)
factors, \citet{ESRB2016} points that an unanticipated rapid green
transition could significantly impact asset values, with
carbon-intensive investments facing declining profitability while
low-carbon assets appreciate. In terms of governance (G) factors,
\citet{ANGINER2018} present compelling evidence that
shareholder-friendly corporate governance structures correlate with
increased bank systemic risk, as measured by Marginal Expected Shortfall
(MES)\citep{ACHARYA2017} and SRISK metrics \citep{BROWNLEES2017}.This
relationship appears particularly pronounced for larger banks,
potentially due to too-big-to-fail considerations, and is amplified in
jurisdictions with more generous financial safety nets. Their findings
derive from analyses of both U.S. banks (1990-2014) and international
banks (2004-2008), suggesting the robustness of this relationship across
different contexts and time periods.

Using qualitative research method, \citet{KUHN2022} investigates
sustainable finance disclosures and initiatives in Germany. They find
that ESG disclosures enable financial institutions to identify exposure
to sectors with high dependency on vulnerable ecosystems. Disclosing
these dependencies improves transparency and helps institutions avoid
investments in activities contributing to ecological degradation such as
deforestation-linked loans. Their study proposes integrating natural
capital valuation into credit risk assessments to account for ``hidden''
risks. Case studies demonstrate that institutions adopting such
frameworks reduce exposure to stranded assets and regulatory penalties.

\subsection{Bank Lending, ESG Risk and Strategic
Alignment}\label{bank-lending-esg-risk-and-strategic-alignment}

Banks have revised their approach to corporate social responsibility
(CSR), placing greater emphasis on managing both direct and indirect
risks associated with lending to firms facing environmental and social
challenges \citep{CARNEVALE2012}. Drawn on the stakeholder theory,
\citet{GOSS2011} examine the link between CSR and the cost of bank
loans. They find that firms with poor CSR performance face higher loan
costs due to the creditor risk perceptions.

\citet{DEMETRIADES2025} investigate lending patterns between major banks
and fossil fuel companies from 2001-2021 using global syndicated loan
data. They find a complex dynamic where banks recognize and price in
climate risks through higher interest rates and shorter loan terms, yet
simultaneously increase loan volumes to brown firms. The findings
suggest that regulatory pressure, particularly in Europe and the US, has
led to more stringent lending policies, though not necessarily reduced
lending volumes.

\citet{BASU2022} report that high-ESG banks complement, rather than
substitute, mortgage lending with community development investments in
poor areas. However, banks are more likely to reject mortgage
applications from these communities, suggesting social washing---using
pro-social rhetoric while limiting actual support. Their findings align
with the legitimacy theory, as banks appear to engage in CSR initiatives
to maintain societal approval rather than genuinely addressing financial
inclusion and social responsibility.

\section{Proposed Stakeholder-Legitimacy
Framework}\label{proposed-stakeholder-legitimacy-framework}

\subsection{The Theory}\label{the-theory}

In the early 2000s, \citet{HOOGHIEMSTRA2000} argues that sustainability
reporting research exhibits diverse and inconsistent findings, primarily
due to the absence of a comprehenCHsive theoretical framework. The
research asserts that legitimacy was the dominant perspective.
\citet{SPENCE2010} identified stakeholder theory as the predominant and
most effective framework for explaining sustainability reporting
practices. They also point out that while many studies mention
stakeholders in general, they do not explicitly reference stakeholder
theory or other theoretical frameworks. Our review confirms their
observations in the literature of banks' ESG disclosure. The majority of
the studies show a preoccupation with stakeholder theory
\citep{GALANT2017, SHEN2016, BUALLAY2021}, legitimacy theory
\citep[e.g.][]{CARNEVALE2014}, and to some extent also institutional
theory \citep{HIGGINS2014, BEBBINGTON2018, CHRISTENSEN2021}. Moreover,
these studies primarily rely on isolated theoretical frameworks rather
than adopting a more holistic approach that integrates multiple
theoretical perspectives on ESG disclosure. In this paper we propose an
integrated Stakeholder-Legitimacy Framework of banks' ESG disclosure,
which synthesizes and extends insights from \citet{CAMPBELL2007}
institutional-economic model and \citet{AGUINIS2012} multilevel CSR
analysis. At its core, the framework positions ESG disclosures as
outcomes of dynamic negotiations between stakeholder pressures and
legitimacy-seeking behaviors, moderated by banks' economic and
institutional contexts.

\begin{figure}

\centering{

\pandocbounded{\includegraphics[keepaspectratio]{flowchart.pdf}}

}

\caption{\label{fig-flowchart}ESG Stakeholder-Legitimacy Disclosure
Framework}

\end{figure}%

In the integrated Stakeholder-Legitimacy framework
(Figure~\ref{fig-flowchart}), stakeholder pressures act as catalysts for
banks' ESG disclosures. According to stakeholder theory, companies
should consider and balance the diverse viewpoints and expectations of
all groups affected by or invested in their operations, rather than
focusing solely on shareholders \citep{BUCHHOLZ2005, LAPLUME2008}.
\citet{FREEMAN1984} suggests that company management should stay attuned
to changing dynamics and trends affecting both their organization's
internal constituents and outside parties. Stakeholder pressures arise
from key external actors, including regulators, investors, and
customers. These actors demand greater transparency and accountability,
motivating banks to provide valuable explanations of how they answer to
the societal call for sustainable business conduct. Banks respond to
these pressures through strategic legitimacy-seeking behaviors. Banks'
ESG disclosures range from symbolic gestures (e.g.~adopting TCFD
guidelines without operational changes) to substantive actions such as
phasing out coal financing. Crucially, legitimacy is not a static
achievement but an ongoing process; banks must continuously adapt to
shifting norms, such as the rise of the TNFD as a biodiversity
benchmark. In the stakeholder-legitimacy context, influential
stakeholders execute greater pressure on companies to explain and
justify their business conduct. Therefore, sustainable reporting and
disclosed ESG information serve as a way for companies to establish and
maintain their legitimacy in the eyes of these stakeholders
\citep{CAMPBELL_D2003}.

The integrated framework further incorporates moderating contexts that
shape ESG disclosures produced by the translation of stakeholder
pressures into banks' legitimacy-seeking behaviors, including economic
conditions such as size and profitability, and institutional context
such as ownership structure and regulatory regimes. These moderators do
not merely correlate with disclosure levels but fundamentally shape
banks' capacity and incentives to respond substantively or symbolically
to legitimacy demands (\citet{CAMPBELL2007}; \citet{AGUINIS2012}).
Below, we analyze each moderator's role within the framework

\subsection{Moderating Context in the Stakeholder-Legitimacy
Framework}\label{moderating-context-in-the-stakeholder-legitimacy-framework}

\subsubsection{Bank (Firm) Size}\label{bank-firm-size}

Bank size (usually proxied by total assets or market capitalization)
moderates the stakeholder-legitimacy dynamic by intensifying both
external pressures and internal capacity. Larger frims inherently face
greater visibility \citep{MEZNAR1995}, attracting heightened scrutiny
from a broader, more heterogeneous stakeholder {]}base (regulators,
NGOs, media) \citep{PELOZA2006}. Therefor, companies with significant
market visibility tend to prioritize enhanced sustainability
initiatives, motivated by the desire to maintain a positive reputation
among their stakeholders \citep{DAMATO2020, ABDI2022}. Simultaneously,
size confers resource advantages: as ESG engagements involve complex
processes and require a large scale to be effective, economies of scale
lower the marginal cost of ESG data collection, verification, and
disclosure processes \citep{YOUN2015, FAVINO2019}. Consequently, larger
banks are structurally positioned to deploy substantive
legitimacy-seeking responses (e.g., comprehensive TCFD-aligned
reporting, sector-wide decarbonization commitments) rather than symbolic
gestures (\citet{DREMPETIC2020}; \citet{LAMANDA2024}). Empirical banking
studies confirm this dual role, showing size positively influences ESG
disclosure breadth and quality (\citet{BUALLAY2019};
\citet{ALAREENI2020}).

\subsubsection{Financial Performance}\label{financial-performance}

Financial performance (e.g., ROA, ROE) moderates the framework by
determining banks' \hspace{0pt}resource slack---the ability to absorb
ESG disclosure costs and invest in substantive legitimacy actions. A
body of literature study the relation between ESG disclosure/performance
and financial performance, mostly focusing on the impact of ESG
activities on financial performance
\citep{FAVINO2019, BUALLAY2019, DAMATO2020, ABDI2022, LAMANDA2024}.
Financial performance as a determinant of ESG disclosure receives less
academic attention. The framework posits that profitable banks possess
greater operational flexibility to reallocate resources toward
stakeholder-responsive ESG initiatives without jeopardizing core
financial objectives \citep{HADDOCK2005, LIU2009, GAMERSCHLAG2011}. This
resource slack enables more substantive legitimacy-seeking behaviours
(e.g., integrating ESG risk into lending criteria, funding green bonds)
aimed at genuine alignment with stakeholder norms \citep{AGUINIS2012}.
Conversely, firms with higher leverage face heightened cost sensitivity
and are less likely to make voluntary environmental disclosures due to
financial constraints and costs \citep{BRAMMER2008}. While
\citet{DYDUCH2017} do not find a significant relationship between
financial performance and sustainability reporting in Polish listed
companies. \citet{YUEN2022} report a U-shaped relationship between ESG
activities and bank profitability, suggesting that only banks exceeding
a threshold can translate stakeholder pressure into costly ESG actions.

\subsubsection{Ownership Structure}\label{ownership-structure}

Ownership structure and regulation can be considered as external
influential factors that are beyond management control. Ownership
structure critically moderates stakeholder pressures and
legitimacy-seeking incentives by altering governance priorities and
power asymmetries. Studies examining different ownership variables have
emerged primarily in recent years. The most investigated ownership
structure include a company's listing status (publicly listed),
institutional shareholding, government involvement (state-owned), and
shareholding concentration. Within the framework, different ownership
types shape how banks prioritize which stakeholder demands merit
substantive responses.

Publicly listed companies, in order to comply with regulations and to be
aligned with industry best practices, as well as to stay attuned to
stakeholder pressure and see societal legitimacy, tend to be actively
engaged in ESG reporting. Publicly listed companies face amplified
legitimacy demands driven by regulatory compliance mandates,
reputational vulnerability, and shareholder expectations. Therefore,
public offering is linked to greater adoption of social reporting
practices \citep{HADDOCK2005}; and publicly listed companies tend to
disclose more sustainability-related information
\citep{GAMERSCHLAG2011}. This compels a baseline of ESG disclosure to
mitigate legitimacy threats. The publicly listed ownership structure was
primarily examined as a variable in studies conducted before 2015.
contemporary studies
\citep[e.g.][]{SAHASRANAMAM2020, AMEEN2022, ALOBAID2024} treat listed
banks as primary disclosure agents (data source), reflecting that
publicly listing embeds banks within accountability networks that
amplify stakeholder salience \footnote{Stakeholder salience refers to
  the degree of priority and attention that managers or organizations
  give to the claims and interests of different stakeholders. See
  \citet{MITCHELL1997}}.

In more recent research, institutional shareholding has emerged as a key
variable of interest, such as \citet{DYCK2019}, \citet{CHEN2020}, and
\citet{AMEEN2022}. \hspace{0pt}Institutional shareholding\hspace{0pt}
transforms the stakeholder-legitimacy dynamic through active,
resource-rich surveillance. Institutional investors are defined as
pension funds, insurance companies, banks, sovereign wealth funds, and
other institutions that manage and invest funds on behalf of their
beneficiaries; they are major players in the global financial markets
\citep{AMEEN2022}. Institutional investors, leveraging their significant
equity stakes and specialized expertise, closely monitor corporate
operations and pressure management, reducing agency costs and aligning
operations with long-term value drivers \citep{SHLEIFER1986}. Due to
their influential position, institutional investors can shape companies'
ESG initiatives to disclose greater ESG-related data, aiming to fulfill
non-financial goals and bolster both their own and the firms' public
image \citep{DYCK2019, GARCIA-SANCHEZ2020}. This escalation of ESG
accountability demands position institutional investors as catalysts for
substantive legitimacy responses. Nevertheless, empirical literature
provides inconsistent findings. Some studies indicate that there is a
positive association between institutional ownership and ESG performance
\citep{DYCK2019, CHEN2020, AMEEN2022, RAIMO2020}. \citet{ZHOU2019}
repots that higher institutional ownership can facilitate firms'
transparency by providing voluntary CSR reports. \citet{ALUCHNA2022}
find a negative association between institutional ownership and
disclosure of the social performance. \citet{QU2007} produces
insignificant results on the relationship between corporate ownership,
including institutional ownership and state ownership, and a company's
CSR engagement. These inconsistencies highlight that the framework's
dependency on contex: institutional priorities (e.g., short-term returns
vs.~ESG integration) moderate their amplification effect on stakeholder
pressures.

While institutional investors' role in ESG practices has garnered
significant attention, the influence of state ownership introduces
unique legitimacy calculus, particularly in balancing public policy
objectives with corporate accountability and transparency.
\citet{TAGESSON2009} find that State-owned corporations disclose more
social information than privately owned corporations. One explanation is
that state-owned corporations face heightened scrutiny, with pressure
from their government stakeholders and public/media attention driving
demands for accountability. These entities appear to have responded by
aligning with stakeholder demands, reflecting compliance with external
expectations: state ownership addresses market failures and improve
social welfare \citep[see][]{STIGLITZ1993}. The intensified social
accountability often compels expansive ESG disclosure on social and
development metrics by state-owned reporting entities
\citep{TAGESSON2009, ZHOU2019, BOSE2018}. However, this response to
social pressure is moderated by the dual legitimacy mandate. State-owned
entities must reconsile commercial viability with political directives.
When these confilict, ESG reporting quality may be compromised
\citep{RAIMO2020}.

Shareholding concentration disrupts stakeholder influence by enabling
controlling owners to insulate legitimacy pursuits. One or several
shareholders that have substantial minority ownership stakes and voting
rights(such as 10 or 20 percent) diminish minority stakeholder voices
through voting control and board dominance \citep{SHLEIFER1997}. This
\hspace{0pt}power asymmetry\hspace{0pt} reduces external accountability
pressures, permitting legitimacy-seeking behaviors that prioritize
private control benefits over public transparency
\citep{SMITH2022, DUCASSY2015}. Controlling shareholders are less
motivated to provide CSR information, with minority shareholders have
less power in CSR disclosure practices \citep{SMITH2022, BORGES2024}.
This manifests empirically as lower CSR performance and less
comprehensive sustainability reporting \citep{LAU2016}.

\subsubsection{Regulatory Environment}\label{regulatory-environment}

In the Stakeholder-Legitimacy framework, the regulatory environment as
an institutional moderator reshapes the translation of stakeholder
pressures to banks' legitimacy-seeking disclosure. It codifies
stakeholder expectations to enforceable norms, alters the cost-benefit
calculus of ESG disclosure, and defines the boundaries between symbolic
compliance and substantive accountability. Regulatory regimes intervene
in the stakeholder-pressure mechanism by \hspace{0pt}institutionalizing
societal expectations.\hspace{0pt}\hspace{0pt} Current voluntary
initialtives of social reporting are insufficienct to achieve coporate
accountability \citep{HESS2007}. When regulators formalise ESG
disclosure requirements (e.g., EU CSRD, national green finance
frameworks), they elevate specific stakeholder concerns to mandated
accountability obligations \citep{QU2007}. This \hspace{0pt}compression
of stakeholder pressures\hspace{0pt} replaces fragmented market demands
with codified legitimacy thresholds. Consequently, regulatory stringency
strengthens both the rigor and reliability of corporate disclosure
practices, fostering greater societal accountability
\citep{CICCHIELLO2023}. The role of regulatory environment is further
acknowledged in \hspace{0pt}reconfiguring information asymmetries and
signalling costs. \hspace{0pt}\citet{DESAI2024} reports that stock
market investors reacts positively to the mandatory ESG disclosure
regulation because mandatory ESG disclosure is considered as a positive
signal by reducing information asymmetry. The compliance threshold
deiminishes the legitimacy benefits of superficial compliance. When ESG
disclosure metrics become audited and standardised, the signalling value
of substantive reporting intensifites while symbolic gestures risk
regulatory penalties and reputational damage
\citep{DESAI2024, BOSE2018}.

Furthermore, regulatory heterogeneity across jurisdictions
\hspace{0pt}moderates stakeholder pressure
translation.\hspace{0pt}\hspace{0pt} Banks in weak institutional
environments may exploit voluntary regimes to pursue legitimacy through
ceremonial conformance, adopting frameworks like PRB while decoupling
operations from disclosures \citep{MANOS2024}. Conversely, stringent
regulations compel operational restructuring. Studies of green banking
mandates demonstrate tighter coupling between policy requirements and
substantive environmental risk integration \citep{BOSE2018}. This
contingency underscores the framework's emphasis on regulatory quality
in determining whether stakeholder pressures catalyse authentic
accountability or reputational arbitrage.

This section presents an integrated Stakeholder-Legitimacy Framework,
tailored to the banking sector, to explain banks' ESG disclosure
practices, synthesizing stakeholder theory, legitimacy theory, and
institutional contexts into a cohesive analytical pathway. The framework
posits that ESG disclosures arise from dynamic interactions between
external stakeholder pressures (e.g., regulators, investors, customers)
and banks' legitimacy-seeking behaviors, moderated by economic factors
(e.g., size, profitability) and institutional conditions (e.g.,
ownership structure, regulatory regimes). By bridging insights from
\citet{CAMPBELL2007} and \citet{AGUINIS2012}, the framework advances a
holistic view of ESG reporting as a negotiated outcome of strategic
adaptation to stakeholder demands and institutional expectations.


  \bibliography{reviewurl.bib}



\end{document}
