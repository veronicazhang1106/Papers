% Options for packages loaded elsewhere
% Options for packages loaded elsewhere
\PassOptionsToPackage{unicode}{hyperref}
\PassOptionsToPackage{hyphens}{url}
\PassOptionsToPackage{dvipsnames,svgnames,x11names}{xcolor}
%
\documentclass[
  authoryear]{elsarticle}
\usepackage{xcolor}
\usepackage[left=3.5cm,right=3.5cm,top=4.5cm,bottom=4.5cm]{geometry}
\usepackage{amsmath,amssymb}
\setcounter{secnumdepth}{5}
\usepackage{iftex}
\ifPDFTeX
  \usepackage[T1]{fontenc}
  \usepackage[utf8]{inputenc}
  \usepackage{textcomp} % provide euro and other symbols
\else % if luatex or xetex
  \usepackage{unicode-math} % this also loads fontspec
  \defaultfontfeatures{Scale=MatchLowercase}
  \defaultfontfeatures[\rmfamily]{Ligatures=TeX,Scale=1}
\fi
\usepackage{lmodern}
\ifPDFTeX\else
  % xetex/luatex font selection
\fi
% Use upquote if available, for straight quotes in verbatim environments
\IfFileExists{upquote.sty}{\usepackage{upquote}}{}
\IfFileExists{microtype.sty}{% use microtype if available
  \usepackage[]{microtype}
  \UseMicrotypeSet[protrusion]{basicmath} % disable protrusion for tt fonts
}{}
\makeatletter
\@ifundefined{KOMAClassName}{% if non-KOMA class
  \IfFileExists{parskip.sty}{%
    \usepackage{parskip}
  }{% else
    \setlength{\parindent}{0pt}
    \setlength{\parskip}{6pt plus 2pt minus 1pt}}
}{% if KOMA class
  \KOMAoptions{parskip=half}}
\makeatother
% Make \paragraph and \subparagraph free-standing
\makeatletter
\ifx\paragraph\undefined\else
  \let\oldparagraph\paragraph
  \renewcommand{\paragraph}{
    \@ifstar
      \xxxParagraphStar
      \xxxParagraphNoStar
  }
  \newcommand{\xxxParagraphStar}[1]{\oldparagraph*{#1}\mbox{}}
  \newcommand{\xxxParagraphNoStar}[1]{\oldparagraph{#1}\mbox{}}
\fi
\ifx\subparagraph\undefined\else
  \let\oldsubparagraph\subparagraph
  \renewcommand{\subparagraph}{
    \@ifstar
      \xxxSubParagraphStar
      \xxxSubParagraphNoStar
  }
  \newcommand{\xxxSubParagraphStar}[1]{\oldsubparagraph*{#1}\mbox{}}
  \newcommand{\xxxSubParagraphNoStar}[1]{\oldsubparagraph{#1}\mbox{}}
\fi
\makeatother


\usepackage{longtable,booktabs,array}
\usepackage{calc} % for calculating minipage widths
% Correct order of tables after \paragraph or \subparagraph
\usepackage{etoolbox}
\makeatletter
\patchcmd\longtable{\par}{\if@noskipsec\mbox{}\fi\par}{}{}
\makeatother
% Allow footnotes in longtable head/foot
\IfFileExists{footnotehyper.sty}{\usepackage{footnotehyper}}{\usepackage{footnote}}
\makesavenoteenv{longtable}
\usepackage{graphicx}
\makeatletter
\newsavebox\pandoc@box
\newcommand*\pandocbounded[1]{% scales image to fit in text height/width
  \sbox\pandoc@box{#1}%
  \Gscale@div\@tempa{\textheight}{\dimexpr\ht\pandoc@box+\dp\pandoc@box\relax}%
  \Gscale@div\@tempb{\linewidth}{\wd\pandoc@box}%
  \ifdim\@tempb\p@<\@tempa\p@\let\@tempa\@tempb\fi% select the smaller of both
  \ifdim\@tempa\p@<\p@\scalebox{\@tempa}{\usebox\pandoc@box}%
  \else\usebox{\pandoc@box}%
  \fi%
}
% Set default figure placement to htbp
\def\fps@figure{htbp}
\makeatother





\setlength{\emergencystretch}{3em} % prevent overfull lines

\providecommand{\tightlist}{%
  \setlength{\itemsep}{0pt}\setlength{\parskip}{0pt}}



 
\usepackage[]{natbib}
\bibliographystyle{elsarticle-harv}


\usepackage{booktabs}
\usepackage{longtable}
\usepackage{array}
\usepackage{multirow}
\usepackage{wrapfig}
\usepackage{float}
\usepackage{colortbl}
\usepackage{pdflscape}
\usepackage{tabu}
\usepackage{threeparttable}
\usepackage{threeparttablex}
\usepackage[normalem]{ulem}
\usepackage{makecell}
\usepackage{xcolor}
\usepackage{graphicx}
\usepackage{adjustbox}
\usepackage{float}
\floatplacement{figure}{H}
\makeatletter
\@ifpackageloaded{caption}{}{\usepackage{caption}}
\AtBeginDocument{%
\ifdefined\contentsname
  \renewcommand*\contentsname{Table of contents}
\else
  \newcommand\contentsname{Table of contents}
\fi
\ifdefined\listfigurename
  \renewcommand*\listfigurename{List of Figures}
\else
  \newcommand\listfigurename{List of Figures}
\fi
\ifdefined\listtablename
  \renewcommand*\listtablename{List of Tables}
\else
  \newcommand\listtablename{List of Tables}
\fi
\ifdefined\figurename
  \renewcommand*\figurename{Figure}
\else
  \newcommand\figurename{Figure}
\fi
\ifdefined\tablename
  \renewcommand*\tablename{Table}
\else
  \newcommand\tablename{Table}
\fi
}
\@ifpackageloaded{float}{}{\usepackage{float}}
\floatstyle{ruled}
\@ifundefined{c@chapter}{\newfloat{codelisting}{h}{lop}}{\newfloat{codelisting}{h}{lop}[chapter]}
\floatname{codelisting}{Listing}
\newcommand*\listoflistings{\listof{codelisting}{List of Listings}}
\makeatother
\makeatletter
\makeatother
\makeatletter
\@ifpackageloaded{caption}{}{\usepackage{caption}}
\@ifpackageloaded{subcaption}{}{\usepackage{subcaption}}
\makeatother
\journal{The British Accounting Review}
\usepackage{bookmark}
\IfFileExists{xurl.sty}{\usepackage{xurl}}{} % add URL line breaks if available
\urlstyle{same}
\hypersetup{
  pdftitle={A Stakeholder-Legitimacy Framework for ESG Disclosure in Banking Under Evolving Regulatory Mandates},
  pdfauthor={Ying (Veronica) Zhang; Lisa Sheenan; Fearghal Kearney; Junyu Zhang; George Hutchinson; Neil Reid; Paul Brereton},
  pdfkeywords={ESG disclosure, IFRS S1 and S2, Stakeholder-Legitimacy
framework, bank risk, ownership, climate change, JEL code M14, M41,
G28},
  colorlinks=true,
  linkcolor={blue},
  filecolor={Maroon},
  citecolor={Blue},
  urlcolor={Blue},
  pdfcreator={LaTeX via pandoc}}


\setlength{\parindent}{6pt}
\begin{document}
\begin{frontmatter}
\title{A Stakeholder-Legitimacy Framework for ESG Disclosure in Banking
Under Evolving Regulatory Mandates}

\author[1]{Ying (Veronica) Zhang%
\corref{cor1}%
\fnref{fn1}}
 \ead{veronica.zhang@qub.ac.uk} 
\author[1,2]{Lisa Sheenan%
%
}

\author[3]{Fearghal Kearney%
%
}

\author[1]{Junyu Zhang%
%
}

\author[1]{George Hutchinson%
%
}

\author[1,3,4]{Neil Reid%
%
}

\author[1]{Paul Brereton%
%
}

\affiliation[1]{organization={Queen's University Belfast, Institute for
Global Food Security (IGFS) School of Biological
Sciences},addressline={Belfast, UK},postcodesep={}}
\affiliation[2]{organization={University College Dublin, UCD Michael
Smurfit Graduate Business School},addressline={Dublin,
Ireland},postcodesep={}}
\affiliation[3]{organization={Queen's University Belfast, Queen's
Business School},addressline={Belfast, UK},postcodesep={}}
\affiliation[4]{organization={Queen's University Belfast, Co-Centre for
Climate + Biodiversity + Water, School of Biological
Sciences},addressline={Belfast, UK},postcodesep={}}

\cortext[cor1]{Corresponding author}
\fntext[fn1]{This work was supported by the BIOFIN-EU Project (Grant
agreement 101135476, UKRI through the UK Horizon Guarantee Fund
(no.10092956) and through the National Environment Research Council
(Grant no. Z503368)}






        
\begin{abstract}
The global financial sector is undergoing a transformative shift as
\hspace{0pt}material risks stemming from environmental and social
challenges, such as\hspace{0pt} biodiversity loss, ecosystem
degradation, \hspace{0pt}and climate change,\hspace{0pt}\hspace{0pt}
underscore the growing importance of Environmental, Social, and
Governance (ESG) factors for economic stability. By reviewing the
literature, this study advances a novel Stakeholder-Legitimacy
Framework, integrating stakeholder theory, legitimacy theory, and
institutional isomorphism to explain how banks' ESG disclosures emerge
from dynamic negotiations between external pressures (e.g., regulators,
investors) and strategic legitimacy-seeking behaviours. The study
reveals that while ESG disclosures mitigate bank-specific risks,
conflicting evidence persists on their financial performance impacts.
Critical gaps include underexplored transmission effects of regulations
through lending networks, inconsistent roles of institutional and state
ownership, and tensions between symbolic compliance and substantive
action.
\end{abstract}





\begin{keyword}
    ESG disclosure \sep IFRS S1 and S2 \sep Stakeholder-Legitimacy
framework \sep bank risk \sep ownership \sep climate change \sep 
    JEL code M14, M41, G28
\end{keyword}
\end{frontmatter}
    

\newpage

\section{Introduction}\label{introduction}

The global financial system is undergoing a seismic shift as
nature-related risks, ranging from biodiversity loss to ecosystem
degradation, emerge as material threats to economic stability
\citep{UNEP2023}. Over the past decade, voluntary sustainability
frameworks like the Task Force on Climate-related Financial Disclosures
(TCFD) have evolved into robust, jurisdictionally mandated systems,
driven by regulatory imperatives to combat greenwashing, harmonise
reporting standards, and align financial flows with planetary
boundaries. Central to this transformation are frameworks such as the
Task Force on Nature-Related Financial Disclosures (TNFD), the
International Sustainability Standards Board (ISSB), and region-specific
regulations like the EU's Corporate Sustainability Reporting Directive
(CSRD) and the UK's Sustainability Disclosure Requirements (SDR) (See
the appendix table). These frameworks reflect a growing consensus that
environmental risks are inextricably linked to financial performance,
necessitating disclosures that integrate double materiality, assessing
both financial risks to firms and firms' impacts on ecosystems.

Banks are uniquely exposed to systemic ESG risks due to their role as
financial intermediaries, warranting specialized regulatory analysis.
Nevertheless, empirical research on how ESG rules influence their risk
exposure, governance, and performance remains fragmented. Critically,
this fragmentation obscures how current ESG disclosures often become
performative tools rather than transparency mechanisms, masking actual
environmental exposures through selective reporting or
``sustainability-washing'' \citep{SMITH2022}. Existing studies
disproportionately focus on non-financial firms
\citep[e.g.][]{YOUN2015, FAVINO2019, BUALLAY2019, DYCK2019}, leaving
critical gaps in understanding banks' unique intermediation role. For
instance, while the TNFD's LEAP methodology and the ISSB's IFRS S1 and
S2 standards are gaining traction, their sector-specific implications
for banking, such as transmission effects through lending networks or
the tension between symbolic compliance and substantive action, are
often misunderstood or inadequately captured in disclosures, leading to
misleading risk assessments.\hspace{0pt}\hspace{0pt} Theoretical
frameworks also remain fragmented, with stakeholder theory, legitimacy
theory, and institutional isomorphism often treated in isolation rather
than as interconnected drivers of ESG disclosure.\footnote{Stakeholder
  theory \citep{FREEMAN1984, MITCHELL1997}, legitimacy theory
  \citep{SUCHMAN1995}, and institutional isomorphism
  \citep{DIMAGGIO1983} have been applied individually to explain ESG
  disclosure motivations. Yet, to our knowledge, \hspace{0pt}no major
  study\hspace{0pt} has synthesised these three frameworks to examine
  their synergetic effects, creating a theoretical silo problem. This
  fragmentation mirrors broader critiques of theoretical dis-integration
  in ESG disclosure \citep{ECCLES2012}.} This theoretical disconnection
perpetuates misleading siloed analyses of bank ESG behaviors, failing to
reflect how these forces dynamically intersect.\hspace{0pt} This paper
addresses these gaps by synthesising empirical evidence, proposing an
integrated theoretical model, and identifying urgent research priorities
at the nexus of finance, regulation, and ecological stewardship.

Current empirical studies show how banks use ESG reporting to navigate
real-world constraints. \citet{THOMPSON2004} examines how banks respond
to environmental stakeholders through reporting practices. Their
findings reveal that bankers value annual reports for assessing
corporate environmental impact, despite their limitations, and express
some interest in expanded environmental disclosures. However, their
focus primarily aims at safeguarding loans, highlighting how disclosures
are often deployed as reputational shields rather than accountability
tools.\hspace{0pt}\hspace{0pt} Their study reflects tensions between
stakeholder demands and bank risk management. Similarly,
\citet{SCHOLTENS2009} analyse how international banks balance different
stakeholder interests through Corporate Social Responsibility (CSR)
reporting, while \citet{CARNEVALE2014} test the direct effects of the
sustainability report on stock price in European markets, demonstrating
that sustainability reports positively influence stock prices by
signaling organizational compliance with evolving norms.
\citet{MURE2021} provide empirical evidence supporting stakeholder
theory and legitimacy theory that ESG disclosure and performance can be
used by banks as a tool to manage their reputation and maintain
stakeholder trust, especially in the face of adverse events.
Collectively, these studies expose a fundamental problem: ESG
disclosures are frequently decoupled from substantive action, engineered
to signal legitimacy to stakeholders while masking underlying
tensions.\hspace{0pt}\hspace{0pt} These findings underscore a critical
gap: existing theories fail to fully explain how banks reconcile
competing stakeholder, legitimacy, and economic imperatives in ESG
implementation. With mandatory frameworks expanding globally, this
disconnect necessitates an integrated conceptual lens now to decode the
strategic interplay between disclosure motivations and
outcomes.\hspace{0pt}\hspace{0pt} This necessitates the integrated
Stakeholder-Legitimacy Framework proposed in Section IV.

In this study, we advance a novel Stakeholder-Legitimacy Framework
\citep{CAMPBELL2007, AGUINIS2012} tailored to banks, synthesising
stakeholder theory, legitimacy theory, and institutional isomorphism.
This model positions ESG disclosures as dynamic outcomes of external
pressures (e.g., regulators, shareholders) and banks' legitimacy-seeking
behaviors, moderated by economic factors (e.g., bank size) and
institutional contexts (e.g., ownership structures). Our synthesis of
extant literature highlights research gaps regarding ownership dynamics,
regulatory efficacy, and systemic risk implications. The proposed
Stakeholder-Legitimacy Framework offers a holistic lens to analyse
banks' ESG disclosures as context-dependent processes shaped by
stakeholder pressures and legitimacy imperatives.

Our study advances sustainable finance research by contributing
significantly across three interconnected
domains.\hspace{0pt}\hspace{0pt} First, we introduce theoretical
innovation through the Stakeholder-Legitimacy Framework, which bridges
siloed theoretical perspectives \citep{ECCLES2012} to holistically
conceptualise banks' ESG disclosures as dynamic, context-dependent
processes. This framework reframes disclosures as outcomes shaped by
external stakeholder pressures (e.g., regulators, activist investors)
and internal legitimacy-seeking behaviors, moderated by institutional
factors like ownership structures. Second, we offer actionable policy
relevance by proposing concrete steps for regulators and policymakers,
such as integrating natural capital valuation into stress testing and
harmonizing ISSB-TNFD reporting interoperability. These recommendations
address concerns about symbolic compliance \citep{SMITH2022} and aim to
align regulatory frameworks with planetary boundary imperatives. Third,
we identify underexplored dimensions of banking-sector ESG research
including empirical inconsistencies regarding ownership dynamics and the
efficacy of mandatory frameworks (e.g., CSRD, EU Taxonomy) in shaping
bank risk profiles and governance \citep{ESRB2016}.

The remainder of the literature review is organized as follows: In
Section II and III, we review the theories underpinning, and the
empirical studies on the relationship between ESG disclosure and the
banking sector, respectively. In Section IV, we propose the integrated
Stakeholder-Legitimacy framework applied to the banking industry and
review the research gaps. Section V is the conclusion section. An
overview of the nature-related financial disclosure frameworks from
different dimensions, as background information of this review, is
provided in the appendix.

\section{Theoretical Foundation}\label{theoretical-foundation}

The relationship between environmental, social, and governance (ESG)
reporting and financial institutions has garnered increasing scholarly
attention, driven by global sustainability challenges. This section
first establishes the conceptual landscape of contemporary ESG
disclosure before synthesising theoretical frameworks that explain how
and why ESG reporting influences financial institutions' operations,
decision-making, and market outcomes.

\subsection{Conceptual Advances in ESG
Disclosure}\label{conceptual-advances-in-esg-disclosure}

Contemporary theoretical concepts and analytical lenses around natural
capital and double materiality offer novel perspectives on integrating
environmental considerations into financial decision-making. ESG
reporting is not merely a compliance exercise but a strategic imperative
for financial institutions. Natural capital concept, as articulated by
\citet{DASGUPTA2021}, emphasises the valuation of biodiversity and
natural capital, with implications for financial modelling
\citep{ATKINSON2014}. \citet{ECCLES2014} provide a theoretical
foundation for integrated reporting, which aligns closely with double
materiality. The concept of ``double-materiality'' was first formally
proposed by the European Commission \citep{EC2019} in Guidelines on
Non-financial Reporting: Supplement on Reporting Climate-related
Information published in June 2019; then was formalised by the
regulation of the Corporate Sustainability Reporting Directive
(CSRD)\citep[see][]{EU_CSRD_2022} as a reporting requirement. It
mandates a company to judge materiality from two perspectives 1) ``the
extent necessary for an understanding of the company's development,
performance and position'' and ``in the broad sense of affecting the
value of the company''; 2) environmental and social impact of the
company's activities on a broad range of stakeholders.

\subsection{Organisational Theories}\label{organisational-theories}

Organisational theories, particularly stakeholder theory and legitimacy
theory, highlight the institutional motivations and external pressures
driving ESG reporting practices. Stakeholder theory, developed by
\citet{FREEMAN1984}, posits that organisations are accountable to a
broad range of stakeholders beyond shareholders, including regulators,
customers (investors), employees, and civil society. Building on this,
\citet{MITCHELL1997} proposed the theory of stakeholder salience
emphasising understanding why and how managers prioritise certain
stakeholders, rather than prescribing whom they should prioritise. Their
theory provides tools to identify stakeholder types via a typology and
explain managerial responses through the concept of salience, which
assesses stakeholders' legitimacy, power, and urgency. This typology
offers a framework for financial institutions to navigate competing
stakeholder interests effectively. \citet{DONALDSON1995} provided
theoretical justification for stakeholder theory, emphasising its
descriptive, instrumental, and normative validity in explaining
corporate behaviour.

Legitimacy theory, foundationally developed by \citet{SUCHMAN1995},
identifies three primary forms of legitimacy\footnote{Legitimacy is a
  generalised perception or assumption that the actions of an entity are
  desirable, proper, or appropriate within some socially constructed
  system of norms, values, beliefs, and definitions
  \citep{GINZEL2000, NEILSEN1987, PERROW1970}.}, pragmatic, moral, and
cognitive, that organisations seek to maintain. \citet{DEEGAN2002}
applied this framework to environmental and social disclosures,
demonstrating how these practices reinforce an organisation's
legitimacy. Firms' legitimacy-seeking behaviours influence the depth and
scope of disclosures, with firms in environmentally sensitive industries
disclosing more to manage public perception. In banking contexts, ESG
disclosures function as legitimacy-management tools to address
stakeholder pressures, as evidenced by sector-specific studies
\citep[see][]{THOMPSON2004, MURE2021, CARNEVALE2014}.

Building on stakeholder theory and legitimacy theory, institutional
isomorphism elucidates the structured mechanisms that drive
organisational conformity. \citet{DIMAGGIO1983} identify three
isomorphic drivers that create organisational homogeneity:

\begin{enumerate}
\def\labelenumi{\arabic{enumi}.}
\item
  \textbf{Coercive isomorphism}: explains how regulatory requirements
  and pressure from stakeholders drive adoption of standardised
  reporting formats.
\item
  \textbf{Mimetic isomorphism}: illuminates why organisations copy
  ``best practice'' reporting methods from industry leaders during
  periods of uncertainty.
\item
  \textbf{Normative isomorphism}: shows how professional networks and
  training lead to similar reporting approaches across organisations.
\end{enumerate}

Their foundation study develops a theoretical framework explaining how
these mechanisms operate within organisational fields. This framework is
valuable in understanding the standardisation and diffusion of reporting
practices across organisations. Empirical research highlights the
importance of institutional pressures, including coercive, normative,
and mimetic forces, in driving firms to conform to ESG expectations.
\citet{DELMAS2004} demonstrate firms' heterogeneous interpretations of
isomorphic forces determine substantive versus symbolic ESG integration.
This divergence mirrors banking practices where identical EU CSRD
mandates yield divergent TNFD implementation depth.
\citet{BEBBINGTON2018} position normative isomorphism as instrumental in
bridging accounting praxis with planetary boundaries, arguing that
professional networks institutionalise ``double materiality'' reporting,
which notes the field's underdevelopment in translating SDGs into bank
capital allocation. \citet{CHRISTENSEN2021} reveal mimetic isomorphism's
unintended consequences: firms under US/EU disclosure mandates
increasingly adopt boilerplate ESG language to signal conformity while
obscuring institution-specific ecological impacts.

\subsection{Economic Theories}\label{economic-theories}

Economic theories, particularly information asymmetry and agency theory,
explain how ESG reporting enhances market efficiency and reduces capital
costs. \citet{JENSEN1976} established the foundational agency theory
framework, explaining the conflicts of interest between principals
(shareholders) and agents (managers). This study lays the groundwork for
understanding how ESG reporting can mitigate agency costs by aligning
stakeholder interests. \citet{AKERLOF1970} work on Information Asymmetry
further explains how unequal information among market participants
creates inefficiencies. \citet{HART1987} provide a foundation for modern
contract theory and influence the theory of the firm and organisational
economics. This paper develops the principal-agent model, and
demonstrate how disclosure requirements can act as screening devices.
\citet{GRAY1996} provide a theoretical framework for understanding the
role of corporate social and environmental reporting in addressing
information asymmetry and enhancing accountability.

\subsection{Integrating Theoretical Frameworks: Addressing Complexity in
Banking}\label{integrating-theoretical-frameworks-addressing-complexity-in-banking}

While each theoretical lens offers valuable insights, the complex and
often contradictory nature of ESG impacts on financial institutions
necessitates an integrated framework. Relying solely on one perspective
provides an incomplete picture. For example, Stakeholder theory explains
why banks respond to pressures but struggles to predict how (substantive
vs.~symbolic). Legitimacy theory explains conformity motives but less so
the strategic heterogeneity observed. Economic theories focus on market
outcomes but may underplay the role of institutional pressures and
legitimacy management unique to highly regulated, systemically important
banks. Furthermore, banks operate under intense regulatory scrutiny,
manage complex information asymmetries (e.g., in lending), face diverse
and powerful stakeholders (regulators, depositors, investors), and are
central to allocating capital towards sustainability goals. This
confluence of factors creates unique tensions. For instance, banks may
need to balance short-term shareholder returns with long-term systemic
risks related to climate change or social inequality.

Therefore, synthesising organisational and economic theories offers a
more robust, multi-dimensional lens. It captures the interplay between
external pressures internal strategic choices, and resulting market
consequences. This integrated approach is essential for understanding
the nuanced and often paradoxical ways ESG reporting shapes the modern
banking landscape.

\section{Extant Empirical Evidence}\label{extant-empirical-evidence}

\subsection{Firm Value and
Performance}\label{firm-value-and-performance}

Stakeholder theories suggest that when organisations incorporate
environmental, social, and governance (ESG) practices into their
long-term strategic planning, they gain competitive advantages by
addressing the needs of various groups - from employees and customers to
government bodies and community members. This approach recognises that
business success is linked to creating value for all stakeholders, not
just shareholders. The stakeholders reward companies with good ESG
practices through investment, consumption and higher productivity
\citep{LI2018}. In contrast, the trade-off hypothesis or traditionalist
view \citep[see][]{FRIEDMAN2007} suggests ESG practices negatively
impact financial performance by increasing costs, harming corporate
performance, and reducing competitive advantage. Scholars argue that
focusing on social and environmental goals diverts managers from
maximising shareholder value. Similarly, satisfying non-shareholder
stakeholders may hinder profit maximization and value creation for
owners and managers\citep{GALANT2017}.

Empirical studies on the relationship between ESG engagement and
financial performance have provided conflicting findings. Driven by
stakeholder theory, \citet{WU2013} find that ESG disclosures improve
bank profitability by attracting ESG-conscious investors. Their findings
attribute to stakeholder pressure for transparency. Drawing from
empirical analysis using MSCI ESG STATS database ratings,
\citet{CORNETT2016} examine the relationship between the US commercial
banks' corporate social responsibility (CSR) practices and financial
performance. The study reveals that larger banks, despite facing
criticism for profit-focused practices leading to the crisis,
demonstrate consistently higher CSR strengths and concerns. Post-2009,
these institutions showed marked improvement with increased CSR
strengths and decreased concerns. Similarly, studies such as
\citet{CARNEVALE2014}, \citet{SHEN2016} and \citet{BUALLAY2021}
demonstrate banks' engagement in CSR activities can improve the
financial performance. \citet{TIAN2023} report that banks with green
credit announcement show better market value and long-term performance.
While \citet{FERRERO2016} find a nonlinear relationship between ESG and
financial performance employing listed companies in Europe. Moreover,
\citet{BUALLAY2023} investigate the effect of sustainability reporting
on bank performance (operational, financial and market) in 7 regions
that include 60 countries over the period 2008-2017. They find the
negative relationship between ESG disclosures and banks' operational
performance (ROA, ROE), and market performance (Tobin's Q). They argue
that sustainability reporting may have negative impact on banks' asset
utilisation, in line with the trad-off theory \citep[see][]{LEE2009}.
\citet{ARAS2024} investigates the SDGs with ESG indicators through a
double materiality perspective for 1888 companies from the OECD
financial institutions. The study shows how commercial banks can
identify and prioritise the SDGs and targets and how sustainable
practices at the corporate level can contribute to achieving these
global goals by adopting a sound approach.

\subsection{Bank Risk}\label{bank-risk}

Despite growing scholarly attention to the link between CSR and bank
performance, empirical research examining its influence on risk exposure
within financial institutions remains limited. The existing literature
primarily focuses on individual bank risk metrics. Research generally
indicates a negative relationship between stronger ESG
performance/disclosure and individual bank risk. Studies across various
regions and time periods have found associations with lower insolvency
risk (Z-scores), reduced leverage and liquidity challenges, and enhanced
stability during financial stress. The length of a bank's ESG reporting
history appears to amplify these stabilizing effects. However, findings
suggest heterogeneity across ESG pillars, with environmental (E) factors
often demonstrating the strongest and most consistent risk-mitigation
outcomes compared to social (S) and governance (G) dimensions.

Academic research on the relationship between systemic financial risk
and ESG factors remains limited, with existing studies examining each of
the three E, S and G pillars respectively. Research highlights potential
vulnerabilities, such as the risk of asset value destabilization from an
unanticipated rapid green transition affecting carbon-intensive sectors.
Conversely, studies focusing on governance factors present a concerning
counterpoint: shareholder-friendly governance structures have been
associated with increased systemic risk, particularly for larger banks,
potentially exacerbated by too-big-to-fail perceptions and generous
safety nets. Qualitative research also offers insights, suggesting that
ESG disclosures, particularly those integrating natural capital
considerations, can help banks identify hidden ecological dependencies
and reduce exposure to stranded assets and regulatory penalties in
sectors like deforestation-linked finance.

The key empirical findings linking ESG disclosure to both individual and
systemic bank risk are summarised in Table~\ref{tbl-literature}.

\begin{table}

\caption{\label{tbl-literature}Summary of Empirical Evidence Linking ESG
Disclosure to Individual and Systemic Bank Risk}

\centering{

\centering\begingroup\fontsize{9}{11}\selectfont

\begin{tabular}[t]{>{\raggedright\arraybackslash}p{2cm}>{\raggedright\arraybackslash}p{5cm}>{\raggedright\arraybackslash}p{5cm}}
\toprule
\textbf{Risk Type} & \textbf{Findings} & \textbf{Implication}\\
\midrule
\cellcolor{gray!10}{Individual Bank Risk} & \cellcolor{gray!10}{Higher ESG scores linked to lower insolvency, leverage, and liquidity risk (e.g Gangi et al., 2019)} & \cellcolor{gray!10}{Supports ESG as a risk management tool}\\
Individual Bank Risk & Longer ESG reporting history enhances risk mitigation (Chiaramonte et al., 2022) & Suggests consistency and time horizon of disclosure matters\\
\cellcolor{gray!10}{Individual Bank Risk} & \cellcolor{gray!10}{Environmental (E) pillar shows strongest risk-reduction effect (Neizert and Petras, 2022)} & \cellcolor{gray!10}{Emphasis importance of environmental reporting}\\
Systemic Financial Risk & Rapid green transition can destabilise assets (ESRB, 2016) & Calls for climate transition scenario analysis\\
\cellcolor{gray!10}{Systemic Financial Risk} & \cellcolor{gray!10}{Shareholder-friendly governance may increase systemic risk (Anginer et al., 2018)} & \cellcolor{gray!10}{Raises concerns for governance-driven risk in large banks}\\
\addlinespace
Qualitative Risk Insights & Natural capital integration may reduce exposure to ecological risks (Kuhn, 2022) & Points to improved credit risk assessment via ESG reporting\\
\bottomrule
\end{tabular}
\endgroup{}

}

\end{table}%

\subsection{Bank Lending, ESG Risk and Strategic
Alignment}\label{bank-lending-esg-risk-and-strategic-alignment}

Banks have revised their approach to corporate social responsibility
(CSR), placing greater emphasis on managing both direct and indirect
risks associated with lending to firms facing environmental and social
challenges \citep{CARNEVALE2012}. Drawn on the stakeholder theory,
\citet{GOSS2011} examine the link between CSR and the cost of bank
loans. They find that firms with poor CSR performance face higher loan
costs due to the creditor risk perceptions.

\citet{DEMETRIADES2025} investigate lending patterns between major banks
and fossil fuel companies from 2001-2021 using global syndicated loan
data. They find a complex dynamic where banks recognize and price in
climate risks through higher interest rates and shorter loan terms, yet
simultaneously increase loan volumes to brown firms. The findings
suggest that regulatory pressure, particularly in Europe and the US, has
led to more stringent lending policies, though not necessarily reduced
lending volumes.

\citet{BASU2022} report that high-ESG banks complement, rather than
substitute, mortgage lending with community development investments in
poor areas. However, banks are more likely to reject mortgage
applications from these communities, suggesting social washing---using
pro-social rhetoric while limiting actual support. Their findings align
with the legitimacy theory, as banks appear to engage in CSR initiatives
to maintain societal approval rather than genuinely addressing financial
inclusion and social responsibility.

\section{Proposed Stakeholder-Legitimacy
Framework}\label{proposed-stakeholder-legitimacy-framework}

In the early 2000s, \citet{HOOGHIEMSTRA2000} argues that sustainability
reporting research exhibits diverse and inconsistent findings, primarily
due to the absence of a comprehensive theoretical framework. The
research asserts that legitimacy was the dominant perspective.
\citet{SPENCE2010} identified stakeholder theory as the predominant and
most effective framework for explaining sustainability reporting
practices. They also point out that while many studies mention
stakeholders in general, they do not explicitly reference stakeholder
theory or other theoretical frameworks. Our review confirms their
observations in the literature of banks' ESG disclosure. The majority of
the studies show a preoccupation with stakeholder theory
\citep{GALANT2017, SHEN2016, BUALLAY2021}, legitimacy theory
\citep[e.g.][]{CARNEVALE2014}, and to some extent also institutional
theory \citep{HIGGINS2014, BEBBINGTON2018, CHRISTENSEN2021}. Moreover,
these studies primarily rely on isolated theoretical frameworks rather
than adopting a more holistic approach that integrates multiple
theoretical perspectives on ESG disclosure. In this paper we propose an
integrated Stakeholder-Legitimacy Framework of banks' ESG disclosure,
which synthesises and extends insights from \citet{CAMPBELL2007}
institutional-economic model and \citet{AGUINIS2012} multilevel CSR
analysis. At its core, the framework positions ESG disclosures as
outcomes of dynamic negotiations between stakeholder pressures and
legitimacy-seeking behaviours, moderated by banks' economic and
institutional contexts.

\begin{figure}

\centering{

\pandocbounded{\includegraphics[keepaspectratio]{flowchart.pdf}}

}

\caption{\label{fig-flowchart}The Stakeholder-Legitimacy Framework for
ESG Disclosure in Banking}

\end{figure}%

In the integrated Stakeholder-Legitimacy framework
(Figure~\ref{fig-flowchart}), stakeholder pressures act as catalysts for
banks' ESG disclosures. According to stakeholder theory, companies
should consider and balance the diverse viewpoints and expectations of
all groups affected by or invested in their operations, rather than
focusing solely on shareholders \citep{BUCHHOLZ2005, LAPLUME2008}.
\citet{FREEMAN1984} suggests that company management should stay attuned
to changing dynamics and trends affecting both their organisation's
internal constituents and outside parties. Stakeholder pressures arise
from key external actors, including regulators, investors, and
customers. These actors demand greater transparency and accountability,
motivating banks to provide valuable explanations of how they answer to
the societal call for sustainable business conduct. Banks respond to
these pressures through strategic legitimacy-seeking behaviours. Banks'
ESG disclosures range from symbolic gestures (e.g.~adopting TCFD
guidelines without operational changes) to substantive actions such as
phasing out coal financing. Crucially, legitimacy is not a static
achievement but an ongoing process; banks must continuously adapt to
shifting norms, such as the rise of the TNFD as a biodiversity
benchmark. In the stakeholder-legitimacy context, influential
stakeholders execute greater pressure on companies to explain and
justify their business conduct. Therefore, sustainable reporting and
disclosed ESG information serve as a way for companies to establish and
maintain their legitimacy in the eyes of these stakeholders
\citep{CAMPBELL_D2003}.

The integrated framework further incorporates moderating contexts that
shape ESG disclosures produced by the translation of stakeholder
pressures into banks' legitimacy-seeking behaviours, including economic
conditions such as size and profitability, and institutional context
such as ownership structure and regulatory regimes. These contexts
fundamentally determine banks' capacity and incentives to pursue
substantive versus symbolic legitimacy responses\hspace{0pt}
\citep{CAMPBELL2007, AGUINIS2012}.

The framework incorporates two primary dimensions of moderating
context.\hspace{0pt}\hspace{0pt} The \hspace{0pt}economic context,
primarily manifested in bank size and financial performance, influences
the resources available and the cost-benefit calculus of ESG engagement.
For example, Larger size typically affords \hspace{0pt}greater
visibility\hspace{0pt} \citep{MEZNAR1995} \hspace{0pt}and resource
advantages\hspace{0pt} \citep{YOUN2015, FAVINO2019}, while stronger
financial performance provides the \hspace{0pt}necessary resource
slack\hspace{0pt} \citep{HADDOCK2005, LIU2009, AGUINIS2012} to absorb
ESG costs. Concurrently, the \hspace{0pt}institutional context,
encompassing factors such as ownership structure (including listing
status, institutional shareholding, state ownership, and ownership
concentration) and the stringency of the regulatory environment,
critically alters \hspace{0pt}governance priorities, stakeholder
salience\hspace{0pt} \citep{MITCHELL1997}, and the
\hspace{0pt}legitimacy thresholds\hspace{0pt} banks must cross to
justify their ESG actions. In particular, the regulatory environment
\hspace{0pt}codifies stakeholder expectations into enforceable
norms\hspace{0pt} \citep{HESS2007, DESAI2024}, defining the boundaries
between symbolic compliance and substantive accountability.
Consequently, these moderating factors do not merely correlate with
disclosure levels; they \hspace{0pt}actively shape the strategic
choices\hspace{0pt} banks make in translating stakeholder pressures into
legitimacy-seeking ESG disclosure behaviours within specific economic
and institutional constraints \citep{CAMPBELL2007, AGUINIS2012}.

\section{Conclusion}\label{conclusion}

The transition from voluntary ESG initiatives to mandatory disclosure
frameworks marks a paradigm shift in how financial institutions
conceptualize and manage nature-related risks. Global frameworks like
the TNFD and ISSB, alongside regional regulations such as the EU
Taxonomy and CSRD, have established rigorous reporting standards that
integrate environmental impacts into financial decision-making. For
banks, these mandates necessitate not only transparency about their own
operations but also heightened scrutiny of borrowers' sustainability
practices \citep[see][]{WANG2023}, creating cascading effects across
economies. Empirical evidence suggests that ESG disclosures can mitigate
bank-specific risks---such as insolvency and liquidity
challenges---while improving market efficiency through reduced
information asymmetry \citep{GANGI2019, SCHOLTENS2019, GANGWANI2024}.
However, conflicting findings on the financial performance-ESG
relationship and the role of ownership structures underscore the need
for contextualized analyses.

To address these complexities, this study advances the
Stakeholder-Legitimacy Framework and tailor it to the banking sector.
This framework synthesises stakeholder theory, legitimacy theory, and
institutional isomorphism to explain how banks' ESG disclosures emerge
from dynamic negotiations between external pressures (e.g., regulators,
investors) and legitimacy-seeking behaviours. It posits that banks
balance competing stakeholder demands---such as investor calls for
transparency and regulatory mandates for biodiversity due
diligence---through strategic disclosures that range from symbolic
compliance (e.g., superficial TNFD adoption) to substantive action
(e.g., divesting from deforestation-linked assets). Crucially, the
framework incorporates moderating factors such as ownership structures
(e.g., institutional ownership, state ownership) and regulatory regimes
(e.g., mandatory CSRD, voluntary IFRS S1 and S2 adoption), offering a
nuanced lens to analyze how economic and institutional contexts shape
disclosure outcomes \citep{CAMPBELL2007, AGUINIS2012}.

While this framework provides a robust analytical tool, important
research gaps warrant attention. Key areas include inconsistent findings
regarding ownership structures' influence on ESG outcomes
\citep[see][]{DYCK2019, ALUCHNA2022}, insufficient exploration of how
mandatory ESG regulations shape bank risk profiles in practice
\citep[see][]{WANG2023}, and banks' dual role as both reporting entities
and sustainability gatekeepers \citep[see][]{SMITH2022}. Additionally,
mixed evidence on the ESG-financial performance nexus highlights the
need for context-specific analyses \citep[see][]{BORGES2024}.

Future research may prioritise three avenues: (1) disentangling the
economic and institutional factors that drive ESG disclosure quality in
banks; (2) evaluating the systemic risk implications of nature-related
regulatory shocks; for instance, assess how nature-related disclosures
influence systemic risk within the banking sector and the wider economy;
and (3) exploring the interaction between stakeholder-legitimacy
framework and institutional isomorphism framework, to assess how
coercive, mimetic, and normative pressures shape banks' ESG disclosure
practices.

The proliferation of nature-related disclosure frameworks in recent
years reveals the shift from voluntary toward mandatory disclosure: from
TNFD's market-driven recommendations to IFRS S1 and S2 disclosure
framework, and mandatory EU reporting standards CSRD. This trend is
driven by several key developments including recognition of material
nature-related risks, global reporting alignment, and integration with
financial reporting. However, accelerating fragmentation creates
compliance complexity, while implementation gaps persist in emerging
economies facing data scarcity and capacity constraints. The fourth
research direction may therefore focus on: harmonization pathways across
frameworks, effectiveness of disclosure mandates in redirecting
conservation financing (e.g.~how metric inconsistencies create
greenwashing vulnerabilities), and tangible impacts on biodiversity loss
reduction---particularly in high-risk lending portfolios.\hspace{0pt}

By focusing on these research avenues through the lens of the
Stakeholder-Legitimacy Framework, scholars and policymakers can enhance
our understanding of ESG disclosure dynamics in the banking sector. This
will not only refine regulatory architectures but also ensure that ESG
disclosures go beyond compliance checklists to drive meaningful
progress.

\section{Appendix: Regional ESG Reporting Framework
Table}\label{appendix-regional-esg-reporting-framework-table}

\newpage

\begin{table}

\caption{\label{tbl-appendix}Appendix: Regional ESG Disclosure
Regulations}

\centering{

\centering\begingroup\fontsize{11}{13}\selectfont

\resizebox{\ifdim\width>\linewidth\linewidth\else\width\fi}{!}{
\begin{tabular}[t]{>{\raggedright\arraybackslash}p{5em}>{\raggedright\arraybackslash}p{8em}>{\raggedleft\arraybackslash}p{5em}>{\raggedright\arraybackslash}p{12em}>{\raggedright\arraybackslash}p{8em}>{\raggedright\arraybackslash}p{10em}>{\raggedright\arraybackslash}p{10em}}
\toprule
\multicolumn{1}{c}{\cellcolor{lightgray}{\textbf{Region}}} & \multicolumn{1}{c}{\cellcolor{lightgray}{\textbf{Regulation}}} & \multicolumn{1}{c}{\cellcolor{lightgray}{\textbf{Year}}} & \multicolumn{1}{c}{\cellcolor{lightgray}{\textbf{Key\_features}}} & \multicolumn{1}{c}{\cellcolor{lightgray}{\textbf{Scope}}} & \multicolumn{1}{c}{\cellcolor{lightgray}{\textbf{Enforcement}}} & \multicolumn{1}{c}{\cellcolor{lightgray}{\textbf{Alignment}}}\\
\midrule
\cellcolor{gray!10}{EU} & \cellcolor{gray!10}{EU Taxonomy Regulation} & \cellcolor{gray!10}{2020} & \cellcolor{gray!10}{a comprehensiver classification system designed to identify environmentally sustainable activities} & \cellcolor{gray!10}{Banks and in-scope entities} & \cellcolor{gray!10}{Failure to comply can result in reputational damage and exclusion from sustainable finance markets} & \cellcolor{gray!10}{EU Taxonomy classification system}\\
EU & Corporate Sustainability Reporting Directive (CSRD) & 2023 & 'double materiality' approach; ESRS & 50,000+ EU/non-EU companies by 2027 & Mandatory for EU/non-EU companies & Aligns with IFRS S1 and S2 standards; TNFD metrics\\
\cellcolor{gray!10}{EU} & \cellcolor{gray!10}{Sustainable Finance Disclosure Regulation (SFDR)} & \cellcolor{gray!10}{2021} & \cellcolor{gray!10}{financial institutions classify investment products based on their sustainability objectives} & \cellcolor{gray!10}{Financial institutions} & \cellcolor{gray!10}{Market exclusion for non-compliance} & \cellcolor{gray!10}{Links to EU Taxonomy}\\
UK & Sustainability Disclosure Requirements (SDR) & 2023 & ISSB-aligned reporting; Transition plans for Paris and GBF; Financed emissions disclosure & Listed companies (from 2026); High-impact sectors (agriculture, mining) & FCA enforcement starting 2026 & Aligns with IFRS S1 and S2; References TNFD recommendations\\
\cellcolor{gray!10}{US} & \cellcolor{gray!10}{SEC Climate Disclosure Rule} & \cellcolor{gray!10}{2024} & \cellcolor{gray!10}{Material climate risk disclosure; Scope 1-3 emissions reporting; Extreme weather cost reporting} & \cellcolor{gray!10}{Public companies} & \cellcolor{gray!10}{Delayed due to legal challenges} & \cellcolor{gray!10}{N/A}\\
\addlinespace
US & Climate Corporate Data Accountability Act (SB 253) & 2026 & Scope 1-3 emissions reporting for \$1B+ revenue firms; Biodiversity risk assessment & Large corporations operating in California & California state law enforcement & Addresses supply chain biodiversity risks\\
\bottomrule
\end{tabular}}
\endgroup{}

}

\end{table}%

\newpage


\bibliography{review.bib}



\end{document}
