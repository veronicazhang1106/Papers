% Options for packages loaded elsewhere
% Options for packages loaded elsewhere
\PassOptionsToPackage{unicode}{hyperref}
\PassOptionsToPackage{hyphens}{url}
\PassOptionsToPackage{dvipsnames,svgnames,x11names}{xcolor}
%
\documentclass[
  letterpaper,
  DIV=11,
  numbers=noendperiod]{scrartcl}
\usepackage{xcolor}
\usepackage{amsmath,amssymb}
\setcounter{secnumdepth}{-\maxdimen} % remove section numbering
\usepackage{iftex}
\ifPDFTeX
  \usepackage[T1]{fontenc}
  \usepackage[utf8]{inputenc}
  \usepackage{textcomp} % provide euro and other symbols
\else % if luatex or xetex
  \usepackage{unicode-math} % this also loads fontspec
  \defaultfontfeatures{Scale=MatchLowercase}
  \defaultfontfeatures[\rmfamily]{Ligatures=TeX,Scale=1}
\fi
\usepackage{lmodern}
\ifPDFTeX\else
  % xetex/luatex font selection
\fi
% Use upquote if available, for straight quotes in verbatim environments
\IfFileExists{upquote.sty}{\usepackage{upquote}}{}
\IfFileExists{microtype.sty}{% use microtype if available
  \usepackage[]{microtype}
  \UseMicrotypeSet[protrusion]{basicmath} % disable protrusion for tt fonts
}{}
\makeatletter
\@ifundefined{KOMAClassName}{% if non-KOMA class
  \IfFileExists{parskip.sty}{%
    \usepackage{parskip}
  }{% else
    \setlength{\parindent}{0pt}
    \setlength{\parskip}{6pt plus 2pt minus 1pt}}
}{% if KOMA class
  \KOMAoptions{parskip=half}}
\makeatother
% Make \paragraph and \subparagraph free-standing
\makeatletter
\ifx\paragraph\undefined\else
  \let\oldparagraph\paragraph
  \renewcommand{\paragraph}{
    \@ifstar
      \xxxParagraphStar
      \xxxParagraphNoStar
  }
  \newcommand{\xxxParagraphStar}[1]{\oldparagraph*{#1}\mbox{}}
  \newcommand{\xxxParagraphNoStar}[1]{\oldparagraph{#1}\mbox{}}
\fi
\ifx\subparagraph\undefined\else
  \let\oldsubparagraph\subparagraph
  \renewcommand{\subparagraph}{
    \@ifstar
      \xxxSubParagraphStar
      \xxxSubParagraphNoStar
  }
  \newcommand{\xxxSubParagraphStar}[1]{\oldsubparagraph*{#1}\mbox{}}
  \newcommand{\xxxSubParagraphNoStar}[1]{\oldsubparagraph{#1}\mbox{}}
\fi
\makeatother


\usepackage{longtable,booktabs,array}
\usepackage{calc} % for calculating minipage widths
% Correct order of tables after \paragraph or \subparagraph
\usepackage{etoolbox}
\makeatletter
\patchcmd\longtable{\par}{\if@noskipsec\mbox{}\fi\par}{}{}
\makeatother
% Allow footnotes in longtable head/foot
\IfFileExists{footnotehyper.sty}{\usepackage{footnotehyper}}{\usepackage{footnote}}
\makesavenoteenv{longtable}
\usepackage{graphicx}
\makeatletter
\newsavebox\pandoc@box
\newcommand*\pandocbounded[1]{% scales image to fit in text height/width
  \sbox\pandoc@box{#1}%
  \Gscale@div\@tempa{\textheight}{\dimexpr\ht\pandoc@box+\dp\pandoc@box\relax}%
  \Gscale@div\@tempb{\linewidth}{\wd\pandoc@box}%
  \ifdim\@tempb\p@<\@tempa\p@\let\@tempa\@tempb\fi% select the smaller of both
  \ifdim\@tempa\p@<\p@\scalebox{\@tempa}{\usebox\pandoc@box}%
  \else\usebox{\pandoc@box}%
  \fi%
}
% Set default figure placement to htbp
\def\fps@figure{htbp}
\makeatother





\setlength{\emergencystretch}{3em} % prevent overfull lines

\providecommand{\tightlist}{%
  \setlength{\itemsep}{0pt}\setlength{\parskip}{0pt}}



 


\usepackage{booktabs}
\usepackage{longtable}
\usepackage{array}
\usepackage{multirow}
\usepackage{wrapfig}
\usepackage{float}
\usepackage{colortbl}
\usepackage{pdflscape}
\usepackage{tabu}
\usepackage{threeparttable}
\usepackage{threeparttablex}
\usepackage[normalem]{ulem}
\usepackage{makecell}
\usepackage{xcolor}
\KOMAoption{captions}{tableheading}
\makeatletter
\@ifpackageloaded{caption}{}{\usepackage{caption}}
\AtBeginDocument{%
\ifdefined\contentsname
  \renewcommand*\contentsname{Table of contents}
\else
  \newcommand\contentsname{Table of contents}
\fi
\ifdefined\listfigurename
  \renewcommand*\listfigurename{List of Figures}
\else
  \newcommand\listfigurename{List of Figures}
\fi
\ifdefined\listtablename
  \renewcommand*\listtablename{List of Tables}
\else
  \newcommand\listtablename{List of Tables}
\fi
\ifdefined\figurename
  \renewcommand*\figurename{Figure}
\else
  \newcommand\figurename{Figure}
\fi
\ifdefined\tablename
  \renewcommand*\tablename{Table}
\else
  \newcommand\tablename{Table}
\fi
}
\@ifpackageloaded{float}{}{\usepackage{float}}
\floatstyle{ruled}
\@ifundefined{c@chapter}{\newfloat{codelisting}{h}{lop}}{\newfloat{codelisting}{h}{lop}[chapter]}
\floatname{codelisting}{Listing}
\newcommand*\listoflistings{\listof{codelisting}{List of Listings}}
\makeatother
\makeatletter
\makeatother
\makeatletter
\@ifpackageloaded{caption}{}{\usepackage{caption}}
\@ifpackageloaded{subcaption}{}{\usepackage{subcaption}}
\makeatother
\usepackage{bookmark}
\IfFileExists{xurl.sty}{\usepackage{xurl}}{} % add URL line breaks if available
\urlstyle{same}
\hypersetup{
  pdftitle={T6.2 Proposal - Event Study},
  pdfauthor={Veronica Zhang; Junyu Zhang},
  colorlinks=true,
  linkcolor={blue},
  filecolor={Maroon},
  citecolor={Blue},
  urlcolor={Blue},
  pdfcreator={LaTeX via pandoc}}


\title{T6.2 Proposal - Event Study}
\author{Veronica Zhang \and Junyu Zhang}
\date{July 21, 2025}
\begin{document}
\maketitle


\subsection{Research Question: Market Reaction to CSRD Events and
Biodiversity Risk
Exposure}\label{research-question-market-reaction-to-csrd-events-and-biodiversity-risk-exposure}

\subsubsection{Hypotheses Formulation}\label{hypotheses-formulation}

\begin{itemize}
\tightlist
\item
  Hypothesis 1: CSRD regulatory events trigger negative abnormal returns
\end{itemize}

The enactment of key CSRD regulatory events (e.g., CSRD final text
adoption, disclosure standards ESRS release) generates statistically
significant negative abnormal returns for European listed firms,
reflecting market anticipation of compliance cost burdens.

\begin{itemize}
\tightlist
\item
  Hypothesis 2: ESG risk exposure (Bloomberg ESG Resource Dependence
  Score) amplifies abnormal (negative) returns
\end{itemize}

The adverse market reaction is significantly stronger for firms with
higher Bloomberg ESG Resource Dependence scores, indicating investor
pricing of biodiversity-related transition risks under more stringent
disclosure requirements.

\subsubsection{Event Window
Specification}\label{event-window-specification}

\begin{table}

\caption{\label{tbl-event}CSRD Event Dates Selection}

\centering{

\centering\begingroup\fontsize{9}{11}\selectfont

\begin{tabular}[t]{>{\raggedright\arraybackslash}p{2cm}>{\raggedright\arraybackslash}p{3cm}>{\raggedright\arraybackslash}p{2cm}>{\raggedright\arraybackslash}p{4cm}}
\toprule
\textbf{Window Type} & \textbf{Time Period} & \textbf{Trading Days} & \textbf{Purpose}\\
\midrule
\textbf{\cellcolor{gray!10}{Estimation Window}} & \cellcolor{gray!10}{2022-03-01 to 2022-12-01} & \cellcolor{gray!10}{{}[-180, -22]} & \cellcolor{gray!10}{Model calibration excluding COP26(2021) policy effects and Ukraine war effects}\\
\textbf{Pre-Event Buffer} & 2022-12-02 to 2022-12-30 & {}[-21, -2] & Monitor anticipatory positioning and exclude information contamination\\
\textbf{\cellcolor{gray!10}{Event Window}} & \cellcolor{gray!10}{2023-01-03 to 2023-01-09} & \cellcolor{gray!10}{{}[-1, +3]} & \cellcolor{gray!10}{Capture market digestion of compliance obligations}\\
\textbf{Post-Event Window} & 2023-01-10 to 2023-04-03 & {}[+5, +60] & Track implementation uncertainty resolution\\
\bottomrule
\end{tabular}
\endgroup{}

}

\end{table}%

\subsection{Statistical Analysis Plan}\label{statistical-analysis-plan}

\subsubsection{Event Study Core Test
(H1)}\label{event-study-core-test-h1}

\begin{itemize}
\tightlist
\item
  \textbf{Daily Abnormal Return (AR) using market method (OLS from
  estimation window)} (event window)
\end{itemize}

\begin{equation}
AR_{it} = R_{it} - (\hat{\alpha_i} + \hat{\beta_i}R_{mt})
\end{equation}

\begin{itemize}
\tightlist
\item
  \textbf{Average Abnormal Return (AAR) (daily, cross-sectional)} (event
  window)
\end{itemize}

Reflect the market's collective reaction on day t within the event
window. AAR is used to analyze the time dynamics of the event's impact,
for example, if the AAR is negative only on day 0, it indicates that the
market reacted quickly to the CSRD release.

\begin{equation}
AAR_{t} = \frac{1}{N} \sum_{i=1}^{N}AR_{it}
\end{equation}

\begin{itemize}
\tightlist
\item
  \textbf{Cumulative Abnormal Return (CAR) per firm} (event window)
\end{itemize}

\begin{equation}
CAR_{\tau_1, \tau_2} = \sum_{t=\tau_1}^{\tau_2}AR_{it}
\end{equation}

\begin{itemize}
\tightlist
\item
  \textbf{Cross-Sectional Average CAR (CACAR)} (event window)
\end{itemize}

The overall market reaction, test the main effect (the overall negative
impact hypothesized in H1)

\begin{equation}
CACAR = \frac{1}{N} \sum_{i=1}^{N}CAR_i
\end{equation}

\begin{itemize}
\tightlist
\item
  \textbf{Step 1: Parametric t-test on CARs (\(CAR_{\tau_1, \tau_2}\))}
\end{itemize}

T-test if the mean of CARs is significantly negative (one-tailed test).
If the p-value is less than 0.05, we reject the null hypothesis (CACAR
is not significantly negative) and support H1: the market reacts
negatively to CSRD events.

\begin{itemize}
\tightlist
\item
  \textbf{Step 2: Non-parametric Corrado test on CACAR (on ARs)}
\end{itemize}

Corrado Rank Test if the p-value is less than 0.05, we reject the null
hypothesis (negative \(AR_{it}\) is not detected) and support H1: the
market reacts negatively to CSRD events.

\subsubsection{Regression Analysis (H2)}\label{regression-analysis-h2}

\begin{equation}
CAR_{i} = \beta_0 + \beta_1ESG_Dep_i + \beta_2ln(Assets_i) + \beta_3ESG_rating_i + \epsilon_{it}
\end{equation}

Variable definitions see Table~\ref{tbl-variable}

\begin{table}

\caption{\label{tbl-variable}Variable Definition}

\centering{

\centering\begingroup\fontsize{9}{11}\selectfont

\begin{tabular}[t]{>{\raggedright\arraybackslash}p{4cm}>{\raggedright\arraybackslash}p{2cm}>{\raggedright\arraybackslash}p{4cm}}
\toprule
\textbf{Term} & \textbf{Symbol} & \textbf{Measurement}\\
\midrule
\textbf{\cellcolor{gray!10}{Dependent Variable}} & \cellcolor{gray!10}{$CAR_i$} & \cellcolor{gray!10}{Cumulative Abnormal Return [-1, +3]}\\
\textbf{Key Predictor} & $ESG\_Dep_i$ & Bloomberg ESG Resource Dependence Score\\
\textbf{\cellcolor{gray!10}{Size Control}} & \cellcolor{gray!10}{$ln(Assets_i)$} & \cellcolor{gray!10}{Ln of total assets}\\
\textbf{ESG Control} & $ESG\_rating_i$ & ESG rating\\
\bottomrule
\end{tabular}
\endgroup{}

}

\end{table}%

Expected output: \(\beta_1\) \textless0 and p\textless0.05, evience
supporting H2: ESG risk exposure (Bloomberg ESG Resource Dependence
Score) amplifies abnormal (negative) returns




\end{document}
