% Options for packages loaded elsewhere
\PassOptionsToPackage{unicode}{hyperref}
\PassOptionsToPackage{hyphens}{url}
\PassOptionsToPackage{dvipsnames,svgnames,x11names}{xcolor}
%
\documentclass[
  letterpaper,
  DIV=11,
  numbers=noendperiod]{scrartcl}

\usepackage{amsmath,amssymb}
\usepackage{iftex}
\ifPDFTeX
  \usepackage[T1]{fontenc}
  \usepackage[utf8]{inputenc}
  \usepackage{textcomp} % provide euro and other symbols
\else % if luatex or xetex
  \usepackage{unicode-math}
  \defaultfontfeatures{Scale=MatchLowercase}
  \defaultfontfeatures[\rmfamily]{Ligatures=TeX,Scale=1}
\fi
\usepackage{lmodern}
\ifPDFTeX\else  
    % xetex/luatex font selection
\fi
% Use upquote if available, for straight quotes in verbatim environments
\IfFileExists{upquote.sty}{\usepackage{upquote}}{}
\IfFileExists{microtype.sty}{% use microtype if available
  \usepackage[]{microtype}
  \UseMicrotypeSet[protrusion]{basicmath} % disable protrusion for tt fonts
}{}
\makeatletter
\@ifundefined{KOMAClassName}{% if non-KOMA class
  \IfFileExists{parskip.sty}{%
    \usepackage{parskip}
  }{% else
    \setlength{\parindent}{0pt}
    \setlength{\parskip}{6pt plus 2pt minus 1pt}}
}{% if KOMA class
  \KOMAoptions{parskip=half}}
\makeatother
\usepackage{xcolor}
\setlength{\emergencystretch}{3em} % prevent overfull lines
\setcounter{secnumdepth}{-\maxdimen} % remove section numbering
% Make \paragraph and \subparagraph free-standing
\makeatletter
\ifx\paragraph\undefined\else
  \let\oldparagraph\paragraph
  \renewcommand{\paragraph}{
    \@ifstar
      \xxxParagraphStar
      \xxxParagraphNoStar
  }
  \newcommand{\xxxParagraphStar}[1]{\oldparagraph*{#1}\mbox{}}
  \newcommand{\xxxParagraphNoStar}[1]{\oldparagraph{#1}\mbox{}}
\fi
\ifx\subparagraph\undefined\else
  \let\oldsubparagraph\subparagraph
  \renewcommand{\subparagraph}{
    \@ifstar
      \xxxSubParagraphStar
      \xxxSubParagraphNoStar
  }
  \newcommand{\xxxSubParagraphStar}[1]{\oldsubparagraph*{#1}\mbox{}}
  \newcommand{\xxxSubParagraphNoStar}[1]{\oldsubparagraph{#1}\mbox{}}
\fi
\makeatother


\providecommand{\tightlist}{%
  \setlength{\itemsep}{0pt}\setlength{\parskip}{0pt}}\usepackage{longtable,booktabs,array}
\usepackage{calc} % for calculating minipage widths
% Correct order of tables after \paragraph or \subparagraph
\usepackage{etoolbox}
\makeatletter
\patchcmd\longtable{\par}{\if@noskipsec\mbox{}\fi\par}{}{}
\makeatother
% Allow footnotes in longtable head/foot
\IfFileExists{footnotehyper.sty}{\usepackage{footnotehyper}}{\usepackage{footnote}}
\makesavenoteenv{longtable}
\usepackage{graphicx}
\makeatletter
\def\maxwidth{\ifdim\Gin@nat@width>\linewidth\linewidth\else\Gin@nat@width\fi}
\def\maxheight{\ifdim\Gin@nat@height>\textheight\textheight\else\Gin@nat@height\fi}
\makeatother
% Scale images if necessary, so that they will not overflow the page
% margins by default, and it is still possible to overwrite the defaults
% using explicit options in \includegraphics[width, height, ...]{}
\setkeys{Gin}{width=\maxwidth,height=\maxheight,keepaspectratio}
% Set default figure placement to htbp
\makeatletter
\def\fps@figure{htbp}
\makeatother
% definitions for citeproc citations
\NewDocumentCommand\citeproctext{}{}
\NewDocumentCommand\citeproc{mm}{%
  \begingroup\def\citeproctext{#2}\cite{#1}\endgroup}
\makeatletter
 % allow citations to break across lines
 \let\@cite@ofmt\@firstofone
 % avoid brackets around text for \cite:
 \def\@biblabel#1{}
 \def\@cite#1#2{{#1\if@tempswa , #2\fi}}
\makeatother
\newlength{\cslhangindent}
\setlength{\cslhangindent}{1.5em}
\newlength{\csllabelwidth}
\setlength{\csllabelwidth}{3em}
\newenvironment{CSLReferences}[2] % #1 hanging-indent, #2 entry-spacing
 {\begin{list}{}{%
  \setlength{\itemindent}{0pt}
  \setlength{\leftmargin}{0pt}
  \setlength{\parsep}{0pt}
  % turn on hanging indent if param 1 is 1
  \ifodd #1
   \setlength{\leftmargin}{\cslhangindent}
   \setlength{\itemindent}{-1\cslhangindent}
  \fi
  % set entry spacing
  \setlength{\itemsep}{#2\baselineskip}}}
 {\end{list}}
\usepackage{calc}
\newcommand{\CSLBlock}[1]{\hfill\break\parbox[t]{\linewidth}{\strut\ignorespaces#1\strut}}
\newcommand{\CSLLeftMargin}[1]{\parbox[t]{\csllabelwidth}{\strut#1\strut}}
\newcommand{\CSLRightInline}[1]{\parbox[t]{\linewidth - \csllabelwidth}{\strut#1\strut}}
\newcommand{\CSLIndent}[1]{\hspace{\cslhangindent}#1}

\KOMAoption{captions}{tableheading}
\makeatletter
\@ifpackageloaded{caption}{}{\usepackage{caption}}
\AtBeginDocument{%
\ifdefined\contentsname
  \renewcommand*\contentsname{Table of contents}
\else
  \newcommand\contentsname{Table of contents}
\fi
\ifdefined\listfigurename
  \renewcommand*\listfigurename{List of Figures}
\else
  \newcommand\listfigurename{List of Figures}
\fi
\ifdefined\listtablename
  \renewcommand*\listtablename{List of Tables}
\else
  \newcommand\listtablename{List of Tables}
\fi
\ifdefined\figurename
  \renewcommand*\figurename{Figure}
\else
  \newcommand\figurename{Figure}
\fi
\ifdefined\tablename
  \renewcommand*\tablename{Table}
\else
  \newcommand\tablename{Table}
\fi
}
\@ifpackageloaded{float}{}{\usepackage{float}}
\floatstyle{ruled}
\@ifundefined{c@chapter}{\newfloat{codelisting}{h}{lop}}{\newfloat{codelisting}{h}{lop}[chapter]}
\floatname{codelisting}{Listing}
\newcommand*\listoflistings{\listof{codelisting}{List of Listings}}
\makeatother
\makeatletter
\makeatother
\makeatletter
\@ifpackageloaded{caption}{}{\usepackage{caption}}
\@ifpackageloaded{subcaption}{}{\usepackage{subcaption}}
\makeatother

\ifLuaTeX
  \usepackage{selnolig}  % disable illegal ligatures
\fi
\usepackage{bookmark}

\IfFileExists{xurl.sty}{\usepackage{xurl}}{} % add URL line breaks if available
\urlstyle{same} % disable monospaced font for URLs
\hypersetup{
  pdftitle={Draft-Literature review},
  pdfauthor={Veronica Zhang},
  colorlinks=true,
  linkcolor={blue},
  filecolor={Maroon},
  citecolor={Blue},
  urlcolor={Blue},
  pdfcreator={LaTeX via pandoc}}


\title{Draft-Literature review}
\usepackage{etoolbox}
\makeatletter
\providecommand{\subtitle}[1]{% add subtitle to \maketitle
  \apptocmd{\@title}{\par {\large #1 \par}}{}{}
}
\makeatother
\subtitle{The Relationship Between Corporate Social Performance (CSP)
and Corporate Financial Performance (CFP)}
\author{Veronica Zhang}
\date{November 7, 2024}

\begin{document}
\maketitle


\subsection{Introduction and
Background}\label{introduction-and-background}

The relationship between Corporate Social Performance (CSP) and
Corporate Financial Performance (CFP) has generated extensive debate
within academic and business communities. Meta-analyses, such as those
by Albertini (2013), Margolis and Walsh (2001), and Orlitzky et al.
(2003), broadly support a positive association between CSP and CFP,
suggesting that firms engaged in socially responsible practices tend to
produce better financial outcomes. However, despite this consensus,
findings across individual studies remain inconclusive. Bruna and
Lahouel (2022) argue that these inconsistent results stem from
theoretical ambiguities, methodological biases (such as issues with
sampling, modeling, and endogeneity), and variations in the measurement
of both CSP and CFP. This section explores the positive, negative, and
context-dependent influences of CSP on CFP, highlighting the diverse
findings and identifying the factors that contribute to the ongoing
debate.

\subsection{Positive Relationship Between CSP and
CFP}\label{positive-relationship-between-csp-and-cfp}

A body of research supports that CSP positively influences CFP, arguing
that socially responsible practices can enhance firm reputation, build
customer loyalty, reduce risk, and create financial value. For example,
Ben Lahouel et al. (2019) and Ben Lahouel et al. (2021) find that firms
with higher CSP experience stronger financial performance, suggesting a
mutually reinforcing relationship between social responsibility and
profitability. Oikonomou et al. (2012) also highlights the
risk-mitigating benefits of CSP, reporting that socially responsible
firms show lower financial risk, while irresponsible behavior associated
with increased risk.

The strategic perspective on CSP is further explored by Deng et al.
(2022). Their paper studies Chinese firms and conclude that CSR aligned
with business strategy, particularly for firms adopting a ``prospector''
approach, positively correlates with CFP. This alignment may strengthen
the CSP-CFP link by embedding social responsibility within core
strategic goals. Similarly, Cho-Min Lin and Wang (2020) document that
CSR, when combined with robust corporate governance practices, improves
a firm's credit rating. Their findings imply that a company's social and
financial health can benefit mutually through responsible governance and
ethical business practices, ultimately leading to higher
creditworthiness and potential financial gains.

\subsection{Neutral and Negative
links}\label{neutral-and-negative-links}

In contrast, some studies challenge the assumption of a positive CSP-CFP
relationship, presenting either a neutral or negative link between CSP
and CFP. McWilliams and Siegel (2000) argue that CSP can sometimes
detract from CFP, especially when social initiatives do not align with a
firm's core competencies or are pursued for external rather than
intrinsic reasons. Kim and Yoo (2022) further explore this perspective,
finding that ownership structures and limited transparency can lead
firms to reduce CSR investment, limiting its impact on financial
returns. Specifically, Kim and Yoo (2022) note that CSR does not
contribute to long-term financial performance in opaque firms,
suggesting that lower market scrutiny might result in exaggerated CSR
efforts that ultimately lack financial merit.

Breuer et al. (2022) provide insights into the role of CEO impact in CSR
decision-making, highlighting how powerful CEOs may over-invest in CSR
to enhance their own reputations rather than the firm's long-term
financial health. They find that in cases of strong CEO influence, CSR
initiatives may be pursued for personal gain, resulting in
overinvestment and a subsequent negative impact on firm value. This
underscores the potential for misalignment between CSR intentions and
financial outcomes when CSR is motivated by management incentives rather
than corporate strategy.

\subsection{Factors Influencing the CSP-CFP
Relationship}\label{factors-influencing-the-csp-cfp-relationship}

The CSP-CFP relationship is influenced by various internal and external
factors, including firm characteristics, governance structures, and
industry context. Previous studies indicate that factors like firm size,
leverage, profitability, and growth potential are critical determinants
that can affect the level and impact of CSR activities (Dupire and
M'Zali, 2018; Waddock and Graves, 1997). Liang et al. (2022) demonstrate
that CSR positively impacts productivity in Chinese firms, particularly
in private and high-tech sectors. These studies highlight the impact of
the factors can vary depending on contextual variables such as industry
type and market conditions.

Governance structures and managerial factors also play crucial roles.
For instance, Breuer et al. (2022) suggest that CEO power can intensify
CSR involvement, sometimes at the cost of financial value when CSR
strategies are pursued for individual reputation. Neubaum and Zahra
(2006) add that long-term ownership by foreign and institutional
investors is positively associated with CSR performance, suggesting that
stakeholder structure can drive or hinder CSP initiatives. Additionally,
Yang et al. (2019) use critical mass theory to argue that female board
directors can influence CSR outcomes, nevertheless, their impact depends
on factors such as age and role, rather than simply their presence.

\subsection{Challenges in Methodologies and
Measurement}\label{challenges-in-methodologies-and-measurement}

Methodological challenges complicate the study of CSP and CFP,
particularly issues related to endogeneity, sample selection, and
measurement inconsistencies. Bruna and Lahouel (2022) provide a critical
review of these challenges, emphasizing that inconsistent findings
across studies may stem from heterogeneity in sampling methods, model
specifications, and the operationalization of CSP. These issues lead to
a fragmented understanding of the CSP-CFP link, as studies using
different samples or measurement approaches generate varied results.
Measurement of CSP varies across studies, creating difficulties in
drawing general conclusions. Inconsistent or imprecise indicators of CSP
and CFP make cross-study comparisons challenging, reducing the
robustness of findings and highlighting a need for standardized
measures.

\subsection{Conclusion and Future Directions
(outlines)}\label{conclusion-and-future-directions-outlines}

\begin{itemize}
\tightlist
\item
  The relationship between CSP and CFP remains a complex and
  context-dependent issue.
\item
  While a substantial amount of evidence supports a positive link,
  findings vary depending on firm-specific characteristics, governance,
  and methodological considerations.
\item
  Future directions

  \begin{itemize}
  \tightlist
  \item
    addressing methodological issues like endogeneity and sample
    selection biases
  \item
    developing standardized measures for CSP and CFP
  \item
    exploring the conditions under which CSP translates into CFP
  \end{itemize}
\end{itemize}

\subsection*{Reference}\label{reference}
\addcontentsline{toc}{subsection}{Reference}

\phantomsection\label{refs}
\begin{CSLReferences}{1}{0}
\bibitem[\citeproctext]{ref-ALBERTINI2013}
Albertini, E., 2013. Does environmental management improve financial
performance? A meta-analytical review. Organization \& Environment 26,
431--457. \url{https://doi.org/10.1177/1086026613510301}

\bibitem[\citeproctext]{ref-BEN_LAHOUEL2019}
Ben Lahouel, B., Gaies, B., Ben Zaied, Y., Jahmane, A., 2019. Accounting
for endogeneity and the dynamics of corporate social -- corporate
financial performance relationship. Journal of Cleaner Production 230,
352--364.
https://doi.org/\url{https://doi.org/10.1016/j.jclepro.2019.04.377}

\bibitem[\citeproctext]{ref-BEN_LAHOUEL2021}
Ben Lahouel, B., Zaied, Y.B., Song, Y., Yang, G., 2021. Corporate social
performance and financial performance relationship: A data envelopment
analysis approach without explicit input. Finance Research Letters 39,
101656. https://doi.org/\url{https://doi.org/10.1016/j.frl.2020.101656}

\bibitem[\citeproctext]{ref-BREUER2021}
Breuer, W., Hass, M., Rosenbach, D.J., 2022. The impact of CEO power and
institutional discretion on CSR investment. Review of Financial
Economics 40, 20--43.
https://doi.org/\url{https://doi.org/10.1002/rfe.1131}

\bibitem[\citeproctext]{ref-BRUNA2022}
Bruna, M.G., Lahouel, B.B., 2022. CSR \& financial performance: Facing
methodological and modeling issues commentary paper to the eponymous FRL
article collection. Finance Research Letters 44, 102036.
https://doi.org/\url{https://doi.org/10.1016/j.frl.2021.102036}

\bibitem[\citeproctext]{ref-LIN2020}
Cho-Min Lin, S.-Y.Y., Clara Chia Sheng Chen, Wang, W.-R., 2020. The
effects of corporate governance on credit ratings: The role of corporate
social responsibility. Emerging Markets Finance and Trade 56,
1093--1112. \url{https://doi.org/10.1080/1540496X.2018.1512486}

\bibitem[\citeproctext]{ref-DENG2022}
Deng, B., Ji, L., Liu, Z., 2022. The effect of strategic corporate
social responsibility on financial performance: Evidence from china.
Emerging Markets Finance and Trade 58, 1726--1739.
\url{https://doi.org/10.1080/1540496X.2021.1925245}

\bibitem[\citeproctext]{ref-DUPIRE2018}
Dupire, M., M'Zali, B., 2018. CSR strategies in response to competitive
pressures. Journal of Business Ethics 148, 603--623.
\url{https://doi.org/10.1007/s10551-015-2981-x}

\bibitem[\citeproctext]{ref-KIM2022}
Kim, S., Yoo, J., 2022. Corporate opacity, corporate social
responsibility, and financial performance. Finance Research Letters 49,
103118. https://doi.org/\url{https://doi.org/10.1016/j.frl.2022.103118}

\bibitem[\citeproctext]{ref-LIANG2022}
Liang, Y., Cai, C., Huang, Y., 2022. The effect of corporate social
responsibility on productivity: Firm-level evidence from chinese listed
companies. Emerging Markets Finance and Trade 58, 3589--3607.
\url{https://doi.org/10.1080/1540496X.2020.1788537}

\bibitem[\citeproctext]{ref-MARGOLIS2001}
Margolis, J.D., Walsh, J.P., 2001. People and profits? The search for a
link between a company's social and financial performance, LEA's
organization and management series. Lawrence Erlbaum Associates,
Publishers, Mahwah, N.J.

\bibitem[\citeproctext]{ref-MCWILLIAMS2000}
McWilliams, A., Siegel, D., 2000. Corporate social responsibility and
financial performance: Correlation or misspecification? Strategic
Management Journal 21, 603--609.
https://doi.org/\url{https://doi.org/10.1002/(SICI)1097-0266(200005)21:5\%3C603::AID-SMJ101\%3E3.0.CO;2-3}

\bibitem[\citeproctext]{ref-NEUBAUM2006}
Neubaum, D.O., Zahra, S.A., 2006. Institutional ownership and corporate
social performance: The moderating effects of investment horizon,
activism, and coordination. Journal of Management 32, 108--131.
\url{https://doi.org/10.1177/0149206305277797}

\bibitem[\citeproctext]{ref-OIKONOMOU2012}
Oikonomou, I., Brooks, C., Pavelin, S., 2012. The impact of corporate
social performance on financial risk and utility: A longitudinal
analysis. Financial Management 41, 483--515.
https://doi.org/\url{https://doi.org/10.1111/j.1755-053X.2012.01190.x}

\bibitem[\citeproctext]{ref-ORLITZKY2003}
Orlitzky, M., Schmidt, F.L., Rynes, S.L., 2003. Corporate social and
financial performance: A meta-analysis. Organization Studies 24,
403--441. \url{https://doi.org/10.1177/0170840603024003910}

\bibitem[\citeproctext]{ref-WADDOCK1997}
Waddock, S.A., Graves, S.B., 1997. THE CORPORATE SOCIAL
PERFORMANCE--FINANCIAL PERFORMANCE LINK. Strategic Management Journal
18, 303--319.
https://doi.org/\url{https://doi.org/10.1002/(SICI)1097-0266(199704)18:4\%3C303::AID-SMJ869\%3E3.0.CO;2-G}

\bibitem[\citeproctext]{ref-YANG2019}
Yang, W., Yang, J., Gao, Z., 2019. Do female board directors promote
corporate social responsibility? An empirical study based on the
critical mass theory. Emerging Markets Finance and Trade 55, 3452--3471.
\url{https://doi.org/10.1080/1540496X.2019.1657402}

\end{CSLReferences}




\end{document}
